\SetPicSubDir{ch-Intro}

\chapter{Introduction}
\vspace{2em}

\section{Motivation}
Quadratic unconstrained binary optimisation (QUBO) is a versatile model that can represent various combinatorial optimisation challenges and has vast applications in both operational research and industry~\abbrevcite{b1}. Most problems that involve choosing a set of binary decisions to optimise an objective function can be expressed as a QUBO problem. Examples include max-cut problems, satisfiability problems, or even problems with more complex constraints, such as the travelling salesman problem \cite{b10}. However, QUBO problems are also NP-hard and are extremely hard to solve efficiently as the search space grows exponentially with the number of binary variables~\cite{b1}. Traditionally, QUBO-solving methods were specialised for a specific problem domain to leverage the domain's unique characteristics, limiting the versatility and portability of these QUBO solvers~\cite{b5}.

The QUBO model shares similarities with the Ising model in Physics, which uses spin variables that are either $-1$ or $+1$ instead of the binary variables in QUBO. The two models can be shown to be equivalent with a change of variable domain. The equivalence enhances the problem domains that could be modelled as QUBO problems and also opens the doors for QUBO problems to be solved by more general quantum-inspired solving methods~\cite{b5}.

With a broad range of classical and quantum-inspired methods available to tackle QUBO problems, it is imperative to consolidate and benchmark existing QUBO solvers to measure their performance when solving different kinds of QUBO problems across various types and sizes of QUBO problems.

\section{Objectives}
In the first section of the report, we aim to measure the performance of 3 quantum-inspired QUBO solvers:
\begin{enumerate}
    \item Quantum Annealing
    \item Neural Network Quantum States
    \item Quantum Approximate Optimization Algorithm
\end{enumerate}
We will also use classical solvers as a baseline for comparison with the quantum-inspired solvers. We will use two classical solvers:
\begin{enumerate}
    \setcounter{enumi}{3}
    \item GUROBI Optimizer
    \item Fixstars Amplify QUBO Solver
\end{enumerate}

%QUBOs are not always effective in practice, though. In particular, they cannot directly:

%Support common real-world situations, such as continuous variables (e.g., prices, commodity flow)
%Support constraints (e.g., max/mins, price cannot exceed a certain value)
%Prove optimality (e.g., you can’t be sure you’ve identified the best solution)

A range of problem types and sizes will be used for the benchmarking which are detailed in \autoref{methodology}. For each solver, we will calculate the probability of success (returning the best solution) and normalised energy (relative solution performance) across QUBO problems of different types and sizes. More details of the solvers and performance metrics are available in \autoref{review} and \autoref{methodology} respectively. The benchmarking results are detailed in \autoref{benchmark}.

In the second section of the report, we will explore how the parameters used for Neural Network Quantum States can affect its performance. We systematically compare two neural network architectures and three different training schedules highlighted in \autoref{nnqsresults}. The results are also detailed in \autoref{nnqsresults}.

The results of the study will allow for a better understanding of the current landscape of quantum-inspired QUBO solvers and how they compare to state-of-the-art classical commercial solvers.