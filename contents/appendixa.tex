\chapter{Curve fitting for NNQS}\label{appendix:curvefitting}
This section will describe how we obtained an analytical form of functions $A(s)$ and $B(s)$ as shown in \autoref{dwaveannealing}. The exact form is not provided in the D-Wave documentation, which only provides a set of $1000$ discrete points \cite{dwavefunctions}. A snapshot of the discrete points is shown in \ref{tab:dwavefunction}.

\begin{table}[!h]
    \centering
    \caption{Discrete points of annealing functions $A(s)$ and $B(s)$}
    \label{tab:dwavefunction}
    \begin{tabular}{cccc}
    \hline
    $s$ & $A(s)$ (GHz) & $B(s)$ (GHz)\\
    \hline
    0 & 9.839417 & 0.259193 \\
    0.001 & 9.746483 & 0.261647 \\
    0.002 & 9.655434 & 0.264113 \\
    ... & ... & ... \\
    0.998 & 0.000000 & 8.449086 \\
    0.999 & 0.000000 & 8.463072 \\
    1.000 & 0.000000 & 8.477070 \\
    \hline
    \end{tabular}
\end{table}

To obtain an analytical form, we utilised the curve fit function from the scipy library. First, we normalise by the maximum value in $B(s)$ as only the ratio of $A$ and $B$ is important. We fit an exponential decay function $a_{A}\cdot e^{-b_{A}\cdot s} + c_{A}$ to $A(s)$ and a quadratic function $a_{B} \cdot s^2 + b_{B} \cdot s + c_{B}$ to $B(s)$. The forms of the functions were chosen with reference to the D-Wave documentation \cite{dwavefunctions}. We obtain fitted constants of $a_{A}=1.11, b_{A} = 7.06, c_A-0.00569, a_B = 0.680, b_B = 0.288, c_B = 0.0305$. The fitted functions are shown in \autoref{fittedequation}.
\begin{figure}[!h]
    \centering
    \includegraphics[width=0.6\linewidth]{images/fitted.jpg}
    \caption{Fitted $A(s)$ and $B(s)$ equations against the normalised annealing fraction $s$}
    \label{fittedequation}
\end{figure}