\chapter{D-Wave quenching}\label{appendix:quenching}
A mid-anneal quench helps us take a snapshot of the current state of the D-Wave annealer by quickly increasing the value of $s$, the anneal fraction. By doing so, we change the system quickly compared to the annealing time, which does not give the system enough time to evolve and freezes the current state. By measuring the distribution of states through repeated sampling, we can peek into the evolution of the wave function of the D-Wave annealing setup and compare it to that of the NNQS.

Practically, we can quench the annealing process in a D-Wave solver by specifying the \textit{anneal\_schedule} parameter with discrete points for (anneal fraction, time) \cite{dwaveadvantage}. For example, a schedule that is $[(0.0,0.0)(10.0,0.5)(11.0,1.0)]$, would mean a normal $10 \mu s$ anneal until $s=0.5$ then a $1 \mu s$ quench where $s$ is increased rapidly until $s=1$. For more information, refer to the D-Wave documentation \cite{dwavequench}.