\chapter{D-wave Quenching}\label{appendix:quenching}
A mid-anneal quench helps us identify the current state of the D-wave annealing by quickly increasing the value of $s$, the anneal fraction. By doing so, we change the system quickly compared to the system which does not give the system enough time to evolve which freezes the current state. By then measuring the distribution of states through repeated sampling, we can peek into the evolution of the wave function of the D-wave annealing setup and compare it to that of the NNQS.

We can quench the annealing process in a D-wave solver by specifying the anneal\_schedule parameter by providing discrete points for (anneal fraction, time) \cite{dwaveadvantage}. For example, a schedule that is $[(0.0,0.0)(10.0,0.5)(11.0,1.0)]$, would mean a $10 \mu s$ anneal until $s=0.5$ then a $1 \mu s$ quench until $s=1$.