\SetPicSubDir{background}

\chapter{Background and Preliminaries}
\vspace{2em}

\section{Quadratic unconstrained binary optimization}
The QUBO problem is defined as
\begin{equation} 
\argmin_{x \in \{0, 1\}^n} x^\intercal Q x
\end{equation}
where $Q \in \boldsymbol{M}_{n\times n}(\mathbb{R})$ is a symmetric/upper triangular square matrix with real coefficients \cite{b1}. In general, the solution does not need to be unique. The density $d$ of a QUBO problem refers to the number of non-zero elements above the main diagonal of $Q$ divided by the total number of elements above the main diagonal. This QUBO model has applications in a wide range of combinatorial optimization problems such as Max-Cut \cite{b2}, number partitioning \cite{b3} and machine scheduling problems \cite{b4}. Problems that are concerned with finding the best set of binary decisions to optimize an objective function can generally be formulated as a QUBO problem \cite{b5}.

Once a problem is expressed in a QUBO format, we can utilize general QUBO-solving methods to efficiently obtain solutions to the original problem without specializing in a particular problem domain \cite{b1}. The following sections will describe multiple examples of QUBO problems and reformulations.

\subsection{Simple problem}
Consider the objective function \begin{equation}
f(x_1, x_2, x_3) = -8x_1 + 6x_2 + 3x_3 - 2 x_1 x_2 + 4 x_2 x_3
\end{equation} where $x_1, x_2, x_3 \in \{0, 1\}$ and we want to maximise $f$ over all possible $x_1$, $x_2$ and $x_3$ (if we wanted to minimise $f$ instead we can simply multiply all the coefficients of $f$ by $-1$). Since $x_i^2 = x_i$ for all $i$, we can redefine the objective function as 

\begin{equation}
f(x) = x^\intercal Q x, x = \begin{bmatrix}
x_1 \\
x_2 \\
x_3 
\end{bmatrix}, 
Q = \begin{bmatrix}
-8 & -1 & 0\\
-1 & 6 & 2\\
0 & 2 & 3
\end{bmatrix}
\end{equation}

The coefficients for $Q_{ii}$ are equal to the coefficient of $x_i$ in the original objective function while the coefficient for $Q_{ij} = Q_{ji}$ where $i \neq j$ is equal to half the original coefficient of $x_ij$ to keep the matrix symmetric. This is a simple QUBO problem and can be solved by enumerating the possible inputs to obtain the optimal solution of $x_1 = 0, x_2 = 1, x_3 = 1$.

\subsection{Knapsack problem}
The knapsack problem is a classic combinatorial optimization problem where we have a knapsack with weight capacity $W > 0$ and $n$ items each having a weight $w_i > 0$, profit $c_i > 0$. The problem is to find the optimal set of items to place in the knapsack to maximize total profit while not exceeding the knapsack weight limit.

Here, we can see an example of how QUBO solvers can also be used to solve combinatorial problems with inequality constraints. Formally, we have

\begin{align}
\max &\sum_{i=1}^n c_i x_i \nonumber\\
\mathrm{st.} &\sum_{i=1}^n w_i x_i \leq W \\
x_i &\in \{0,1\}^n \nonumber
\end{align}

To include the inequality constraint into our QUBO formulation, we introduce slack variables, $y_i$'s, which turn the inequality constraint into an equality constraint \cite{b6}.

\begin{align}
\max &\sum_{i=1}^n c_i x_i \nonumber\\
\mathrm{st.} &\sum_{i=1}^n w_i x_i = \sum_{j=1}^W jy_j\\
&\sum_{j=1}^W y_j = 1 \nonumber\\
x &\in \{0,1\}^n, y \in \{0,1\}^W \nonumber
\end{align}

With any penalty parameters $C_1 > \sum_{i=1}^n c_i$ and $C_2 > \sum_{i=0}^n c_i$, we can formulate the following QUBO problem 

\begin{align}
    \max f(x, y) &\coloneqq \sum_{i=1}^n c_i x_i - C_1 P_w^2 - C_2 P_n^2 \\
    P_w &\coloneqq \sum_{i=1}^n w_i x_i - \sum_{j=1}^W jy_j \\
    P_n &\coloneqq 1 - \sum_{j=1}^W y_j \\
    x &\in \{0,1\}^n, y \in \{0,1\}^W \nonumber
\end{align}

Since the penalty parameters are chosen to be larger than $\sum_{i=1}^n c_i$, the optimal solution to the QUBO problem must have $P_w = P_n = 0$. Hence, ensuring that exactly one of the $y_i=1$ and the total weight is below the knapsack capacity. Since the optimal solution to the original knapsack must also have a total weight below the knapsack capacity, the value of $x$ that solves the QUBO must also solve the original knapsack problem.

\subsection{Practical applications}
There are many practical scenarios where QUBO formulations can be utilized. 
\begin{itemize}
    \item \cite{b7} uses real-world data of the location of DB Schenker shipping hubs in Europe to solve the shipment rerouting problem which aims to reduce the total distance traveled to fulfill a set of shipments.
    \item \cite{b8} uses a QUBO formulation to solve portfolio optimization problems using real-world stock data sets of the New York Stock Exchange.
    \item \cite{b9} formulates an image binarization method as a QUBO problem where the objective is to segment an image into its foreground and background which has further possible medical applications to improve x-ray imaging.
\end{itemize}

\section{Ising model}
The Ising model in Physics can be thought of as a model of a magnet \cite{b11}. In the general Ising model, a magnet consists of $n$ molecules that are constrained to lie on the sites of a regular lattice \cite{b11}. Each molecule $i$ can be treated as a microscopic magnet that points along some axis and can have a 'spin' ($s_i)$ that is either $+1$ (parallel to the axis) or $-1$ (anti-parallel to the axis). With $N$ particles, the system can then have $2^N$ states, each corresponding to a configuration of the individual molecule spins.

\subsection{Ising Hamiltonian}
In quantum mechanics, the Hamiltonian is an operator that represents the total energy of a system. In the Ising model, the possible energies of the system are exactly the eigenvalues of its Hamiltonian and the corresponding eigenvectors are the possible states that result in the energy level \cite{b21}. The Hamiltonian of the Ising model has two components --- the external field term (characterized by $h$) and the interaction term between molecules (characterized by $J$) \cite{b10}. 

\begin{equation}
    H(s) = H(s_1, ... , s_n) = -\sum_{i < j} J_{ij}s_i s_j - \sum_{i=1}^N h_i s_i
\end{equation}

To find the ground state (lowest energy state) of the Ising model, we then have to solve for $\argmin H(s)$ for $s \in \{0, 1\}^N$. This is equivalent to a corresponding QUBO problem up to a change in basis. The equivalence of the QUBO problem and the Ising model is one of the most significant applications of QUBO \cite{b5}.

\subsection{Converting Ising to QUBO}
Given the Hamiltonian of an Ising problem, we use the conversion $s_i = 2x_i - 1, x_i \in \{0, 1\}$,

\begin{align}
    H(s) &= -\sum_{i < j} J_{ij}s_i s_j - \sum_{i=1}^N h_i s_i \nonumber\\
    &= -\sum_{i < j} J_{ij}(2x_i - 1) (2x_j - 1) - \sum_{i=1}^N h_i (2x_i - 1) \nonumber\\
    &= -\sum_{i < j} J_{ij}(4x_i x_j - 2x_i - 2x_j + 1) - \sum_{i=1}^N (2h_i x_i - h_i) \nonumber\\
    &\text{Group constants into $k$} \nonumber \\
    &\text{Let $a_i = \sum_{j \neq i} 2J_{\min(i,j)\max(i,j)} + 2h_i$} \nonumber \\
    &= -\sum_{i < j} 4J_{ij}x_i x_j - \sum_{i=0}^N a_{i}x_i + k \nonumber
\end{align}

Removing the constant $k$ which is irrelevant for optimization, we can reformulate the Ising model as a QUBO model with matrix $Q$ such that $Q_{ii} = -a_i$ and $Q_{ij} = Q_{ji} = -2J_{ij}$ for $i \neq j$. The solution to the QUBO problem can then be converted to a solution for the ground state of the Ising model using $s_i = 2x_i - 1$. However, note that the optimal objective function value may be different due to the constant $k$.

\subsection{Converting QUBO to Ising}

Given a QUBO problem with matrix $Q$, we can use the conversion, $x_i = \frac{s_i + 1}{2}, s_i \in \{-1, 1\}$. The objective function $f(x)$ of the QUBO problem can be expressed as

\begin{align}
    f(x) &= f(x_1, ..., x_n) \nonumber\\
    &= \sum_{i < j} Q_{ij}(x_i x_j) + \sum_{i=1}^N Q_{ii} x_i \nonumber \\
    &= \sum_{i < j} \frac{1}{4} Q_{ij}(s_i + 1)(s_j + 1) + \sum_{i=1}^N \frac{1}{2} Q_{ii} (s_i + 1) \nonumber \\
    &\text{Group constants into $k$} \nonumber \\
    &\text{Let $a_i = \sum_{j \neq i} \frac{1}{4}Q_{\min(i,j)\max(i,j)} + \frac{1}{2}Q_{ii}$} \nonumber \\
    &= \sum_{i < j} \frac{1}{4} Q_{ij}s_i s_j + \sum_{i=1}^N a_i s_i + k \nonumber \\
\end{align}

Removing the constant $k$ which is irrelevant for optimization, we can reformulate the QUBO problem as an Ising model with $h_i = -a_i$ and $J_{ij} = -\frac{1}{4}Q_{ij}$ for $i \neq j$. Similarly, the ground state for the Ising model can be converted to a solution for the QUBO problem and the optimal objective function value may be different due to the constant $k$.