\chapter{Conclusion}

This chapter summarises the contributions and limitations of our study. There are also recommendations for further work that could be investigated by future projects.

\section{Contributions}
In this study, we have benchmarked $5$ QUBO solvers:
\begin{enumerate}
    \item D-Wave Quantum Annealing
    \item Neural Network Quantum States (NNQS)
    \item Quantum Approximate Optimization Algorithm (QAOA)
    \item GUROBI Optimizer
    \item Fixstars Amplify QUBO Solver
\end{enumerate}
$3$ datasets were used which comprised of $3$ types of combinatorial optimization problems:
\begin{enumerate}
    \item Not-all-equal 3-satisfiability (NAE3SAT)
    \item Max-cut
    \item Sherrington-Kirkpatrick model (SK model)
\end{enumerate}
Among the quantum-inspired solvers, our results show that the D-wave and NNQS solvers achieve the best performance. The NNQS solver is generally better than the D-wave solver except for the SK model dataset. The QAOA solver achieves generally poor performance across datasets and is not comparable due to the lack of large-scale gate-based quantum computers. All 3 quantum-inspired solvers underperform the 2 classical solvers with the simulated-annealing-based Fixstar Amplify QOBO solver achieving the best performance out of all solvers for all datasets.

We also investigated the NNQS solver with different architectures and training algorithms. The Restricted Boltzmann Machine (RBM) and the Multilayer Perceptron (MLP) were used as the underlying neural networks along with $3$ different training algorithms---progressive, direct, and continuous. We found that using the RBM with a continuous training scheme gives the best performance across all datasets.

The main limitation of the study was for the QAOA solver which was from the limited access to a working gate-based quantum computer from IBMQ. This meant that we had to use the simulator instead which could only handle up to $30$ variables. However, as the QAOA simulator does not model quantum noise and the results from the QAOA solver for small problems are not promising, QAOA on a real quantum device would perform even worse for larger problems and is thus not interesting to test in the current NISQ era. The other significant limitation was that the NNQS exploration dataset was relatively smaller due to runtime constraints. However, as there were still $60$ problems for each solver, the standard error for performance was small.

\section{Future work}
Future studies can explore more QUBO problem types and attempt to classify classes of problems that are difficult for annealing-based solvers such as QA and SA but easier for other solvers such as QAOA. When gate-based quantum computers are readily available, the QAOA solver with higher $p > 1$ values could be benchmarked which should give better performance.

For further work with NNQS, future studies could investigate if more modern machine learning models such as Graph Neural Networks or Attention-based Neural Networks can be used as the underlying architecture for NNQS and whether they provide better performance.

Future work could also investigate whether NNQS closely approximates the wave function of a D-wave solver in the quantum annealing process. This could be done by conducting a quench of the D-wave annealing process in order to take a snapshot of the intermediate state. Some initial work is detailed in \autoref{appendix:quenching} but has not been included in the main report as the results are not substantial.