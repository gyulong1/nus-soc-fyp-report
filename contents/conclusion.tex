\chapter{Conclusion}

This chapter summarises the contributions and limitations of our study. There are also recommendations for further work that could be investigated by future projects.

\section{Contributions}
In this study, we have benchmarked $5$ QUBO solvers:
\begin{enumerate}
    \item D-Wave Quantum Annealing
    \item Neural Network Quantum States (NNQS)
    \item Quantum Approximate Optimization Algorithm (QAOA)
    \item GUROBI Optimizer
    \item Fixstars Amplify QUBO Solver
\end{enumerate}
The datasets used comprised of $3$ types of combinatorial optimization problems:
\begin{enumerate}
    \item Not-all-equal 3-satisfiability (NAE3SAT)
    \item Max-cut
    \item Sherrington-Kirkpatrick model (SK model)
\end{enumerate}

\section{Future work}
Future studies can explore more QUBO problem types and attempt to classify classes of problems that are difficult for annealing-based solvers such as QA and SA but easier for other solvers such as QAOA. When gate-based quantum computers are readily available, the QAOA solver with higher $p > 1$ values could be benchmarked which should give better performance.

For further work with NNQS, future studies could investigate if more modern machine learning models such as Graph Neural Networks or Attention-based Neural Networks can be used as the underlying architecture for NNQS and whether they provide better performance.

Future work could also investigate whether NNQS closely approximates the wave function of a D-wave solver in the quantum annealing process. This could be done by conducting a quench of the D-wave annealing process in order to take a snapshot of the intermediate state. Some initial work is detailed in \autoref{appendix:quenching} but has not been included in the main report as the results are not substantial.