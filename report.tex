\documentclass[fyp,12pt]{socreport}

% Some generic packets.
\usepackage{color, colortbl}
\usepackage{url}
\usepackage{graphicx}
\usepackage{caption}
\usepackage{subcaption}
\usepackage{pgfplots}
\usepackage{tabularx}
\usepackage{multirow}
\usepackage{listings}
\usepackage{fullpage}
\usepackage{amsthm}
\usepackage{amssymb}
\usepackage{hyperref}
\usepackage{algorithm}
\usepackage[noend]{algpseudocode}
\usepackage[activate={true,nocompatibility},final,tracking=true,kerning=true,spacing=true,factor=1100,stretch=10,shrink=10]{microtype}
\usepackage{booktabs}
\usepackage{siunitx}



\theoremstyle{definition}
\newtheorem{defn}{Definition} \newcommand{\defnautorefname}{Definition}
\theoremstyle{plain}
\newtheorem{rul}{Rule} \newcommand{\rulautorefname}{Rule}
\newtheorem{thm}{Theorem} \newcommand{\thmautorefname}{Theorem}
\newtheorem{lemma}{Lemma} \newcommand{\lemmaautorefname}{Lemma}
\newtheorem{coroll}{Corollary} \newcommand{\corollautorefname}{Corollary}
\newcommand{\probP}{\text{I\kern-0.15em P}}
\pgfplotsset{width=10cm,compat=1.9}

% Sets the root path to look for all images.
\graphicspath{{images/}}

% Sets default options for listings.
\renewcommand\lstlistlistingname{List of Listings}
\newcommand*\lstinputpath[1]{\lstset{inputpath=#1}}
\lstinputpath{listings}
\lstset{frame=single, tabsize=2, captionpos=b}
\newcommand{\itab}[1]{\hspace{0em}\rlap{#1}}
\newcommand{\tab}[1]{\hspace{.11\textwidth}\rlap{#1}}
\newcommand{\algorithmautorefname}{Algorithm}

\usepackage{mathtools}
\DeclarePairedDelimiter{\ceil}{\lceil}{\rceil}
\DeclarePairedDelimiter{\floor}{\lfloor}{\rfloor}
\DeclarePairedDelimiter{\vbar}{\vert}{\vert}
\DeclarePairedDelimiter{\vbarbar}{\Vert}{\Vert}
\DeclarePairedDelimiter{\parenLR}{\lparen}{\rparen}
\DeclarePairedDelimiter{\brackLR}{\lbrack}{\rbrack}
\DeclarePairedDelimiter{\braceLR}{\lbrace}{\rbrace}
\DeclareMathOperator*{\argmin}{arg\,min}
\DeclareMathOperator*{\argmax}{arg\,max}

%%% Handy commands for image/figure insertion
\newcommand*{\RootPicDir}{pic}
\newcommand*{\PicDir}{\RootPicDir}
\newcommand*{\ResetPicDir}{\renewcommand*{\PicDir}{\RootPicDir}}
\newcommand*{\SetPicSubDir}[1]{\renewcommand*{\PicDir}{\RootPicDir/#1}}
\newcommand*{\Pic}[2]{\PicDir/#2.#1}

\newcommand*{\RootExpDir}{exp}
\newcommand*{\ExpDir}{\RootExpDir}
\newcommand*{\ResetExpDir}{\renewcommand*{\ExpDir}{\RootExpDir}}
\newcommand*{\SetExpSubDir}[1]{\renewcommand*{\ExpDir}{\RootExpDir/#1}}
\newcommand*{\Exp}[2]{\ExpDir/#2.#1}

\newcommand*{\BeforeCaptionVSpace}{1ex}
\newcommand*{\BeforeSubCaptionVSpace}{0.75ex}

\begin{document}
\pagenumbering{roman}

% Replace as necessary
\title{Data-driven and Physics-inspired Machine Learning: Benchmarking quantum-inspired quadratic unconstrained binary optimisation solvers}
\author{Guo Yulong}
\projyear{2023/2024}
\projnumber{H0811550}
\advisor{Assoc Prof St\'ephane Bressan}
\deliverables{
    \item \itab{Report:} \tab{1 Volume}
}
\maketitle

% Replace as necessary
\begin{abstract}
The quadratic unconstrained binary optimisation (QUBO) problem represents a cornerstone in combinatorial optimisation that encapsulates various scientific and industrial optimisation problems. However, the QUBO problem is NP-hard and is traditionally extremely difficult to solve efficiently due to the exponentially increasing search space. Our work aims to measure the performance of different quantum-inspired QUBO solvers that utilise quantum effects to search for optimal solutions. We benchmark the D-Wave Quantum Annealer, Neural Network Quantum States (NNQS), and Quantum Approximate Optimization Algorithm (QAOA) solvers with classical commercial QUBO solvers. We use a dataset with various combinatorial problems---not-all-equal 3-satisfiability, max-cut, and the Sherrington-Kirkpatrick model (SK model), and the solvers will be evaluated based on the average normalised energy and success probability. Our results show that the D-wave and NNQS solvers outperform the QAOA solver but cannot match the performance of commercial QUBO solvers. However, the D-wave solver shows promising performance compared to the GUROBI solver for some problem sizes when we match the solver runtimes. We further investigate the performance of the NNQS solver when using different architectures and training algorithms. Our results show that the NNQS with a Restricted Boltzmann Machine and a continuous training algorithm perform best across all datasets.

\begin{descriptors}
    \item \itab{10002950.10003714.10003716.10011136 Discrete optimization}
    %\item \itab{10002950.10003714.10003716.10011136 Mathematics of computing~Discrete optimization}
    \item \itab{10010405.10010432.10010441 Applied computing~Physics}
    \item \itab{10010147.10010257.10010293.10010294 Neural networks}

\end{descriptors}
\begin{keywords}
    Quadratic unconstrained binary optimisation, Ising model, Quantum annealing, Neural network quantum states, Quantum Approximate Optimization Algorithm
\end{keywords}

% Replace/Delete as necessary
\begin{implement}
    \item{Linux, GPUs, Tesla A100, Python, Tensorflow, Dimod, Qiskit}
\end{implement}

\end{abstract}

% Replace as necessary
\begin{acknowledgement}
I would like to thank my friends and family for their invaluable support during the project. I would also like to thank my project advisor, Associate Professor St\'ephane Bressan, research fellow Lu Jianlong, and past and present research group members for their invaluable guidance and support.
\end{acknowledgement}

\listoffigures % Remove if no figures
\listoftables % Remove if no tables
\tableofcontents

% Actual contents in the contents folder.
% Include additional tex files here.
\SetPicSubDir{ch-Intro}

\chapter{Introduction}
\vspace{2em}

\section{Motivation}
The quadratic unconstrained binary optimization (QUBO) problem arises in many types of combinatorial optimization tasks and has vast applications in both operational research and industry~\cite{b1}. Problems that involve choosing a set of binary decisions to optimise an objective function can usually be expressed as a QUBO problem. However, QUBO problems are NP-hard and are extremely hard to solve efficiently~\cite{barahona1982computational,b1}. Traditionally, QUBO-solving methods are specialized for a specific problem domain in order to leverage the unique characteristics of each problem domain, limiting the versatility and portability of QUBO-solving methods~\cite{b5}.

The QUBO problem shares similarities with the Ising model in Physics, where variables are expressed as either $-1$ or $1$, instead of the binary variables used in QUBO. The two models turn out to be equivalent and the correspondence between the QUBO problem and the Ising model has opened the doors to solving QUBO problems with quantum computational methods~\cite{b5}. The equivalence not only enhances the problem domains that could be modelled as QUBO problems but also allows for the QUBO problem to be solved by more general quantum-based solving methods~\cite{b5}.

With a broad range of classical and quantum-based methods available to tackle QUBO problems, it is imperative to consolidate and benchmark existing QUBO-solving methods to highlight their strengths and weaknesses for solving different kinds of QUBO problems.

\section{Objectives}
In the first section of the report, we aim to measure the performance of 3 quantum-based QUBO-solving methods:
\begin{itemize}
    \item Quantum Annealing
    \item Neural Network Quantum States
    \item Quantum Approximate Optimization Algorithm
\end{itemize}
For comparison with quantum-based methods, we will also use classical solvers as a baseline. We will explore 2 classical solvers:
\begin{itemize}
    \item Fixstars Amplify QUBO Solver
    \item GUROBI Optimizer
\end{itemize}

%QUBOs are not always effective in practice, though. In particular, they cannot directly:

%Support common real-world situations, such as continuous variables (e.g., prices, commodity flow)
%Support constraints (e.g., max/mins, price cannot exceed a certain value)
%Prove optimality (e.g., you can’t be sure you’ve identified the best solution)

For each solver, we will measure the success rate (probability of solving optimally) and effectiveness (performance of candidate solution) across QUBO problems of different types and sizes. More details of the solvers and performance metrics are available in \autoref{review} and \autoref{methodology}. The results of the study will allow us to understand the current state of quantum-based QUBO-solving methods and how they compare to classical solvers. 

In the second section of the report, we will further explore how Neural Network Quantum States can be used to simulate the quantum annealing process to solve QUBO problems. We first sample intermediate states from both the NNQS and a physical quantum annealer to compare the annealing process that is being simulated in the NNQS. We will also conduct a systematic comparison of two different training algorithms highlighted in \autoref{methodology} for Neural Network Quantum States and extract any physical insights they provide which will bring us one step closer in harnessing its full potential.

\SetPicSubDir{background}

\chapter{Background and Preliminaries}
\vspace{2em}

\section{Quadratic unconstrained binary optimization}
The QUBO problem is defined as
\begin{equation} 
\argmin_{x \in \{0, 1\}^n} x^\intercal Q x
\end{equation}
where $Q \in \boldsymbol{M}_{n\times n}(\mathbb{R})$ is a symmetric/upper triangular square matrix with real coefficients \cite{b1}. In general, the solution does not need to be unique. The density $d$ of a QUBO problem refers to the number of non-zero elements above the main diagonal of $Q$ divided by the total number of elements above the main diagonal. This QUBO model has applications in a wide range of combinatorial optimization problems such as Max-Cut \cite{b2}, number partitioning \cite{b3} and machine scheduling problems \cite{b4}. Problems that are concerned with finding the best set of binary decisions to optimize an objective function can generally be formulated as a QUBO problem \cite{b5}.

Once a problem is expressed in a QUBO format, we can utilize general QUBO-solving methods to efficiently obtain solutions to the original problem without specializing in a particular problem domain \cite{b1}. The following sections will describe multiple examples of QUBO problems and reformulations.

\subsection{Simple problem}
Consider the objective function \begin{equation}
f(x_1, x_2, x_3) = -8x_1 + 6x_2 + 3x_3 - 2 x_1 x_2 + 4 x_2 x_3
\end{equation} where $x_1, x_2, x_3 \in \{0, 1\}$ and we want to maximise $f$ over all possible $x_1$, $x_2$ and $x_3$ (if we wanted to minimise $f$ instead we can simply multiply all the coefficients of $f$ by $-1$). Since $x_i^2 = x_i$ for all $i$, we can redefine the objective function as 

\begin{equation}
f(x) = x^\intercal Q x, x = \begin{bmatrix}
x_1 \\
x_2 \\
x_3 
\end{bmatrix}, 
Q = \begin{bmatrix}
-8 & -1 & 0\\
-1 & 6 & 2\\
0 & 2 & 3
\end{bmatrix}
\end{equation}

The coefficients for $Q_{ii}$ are equal to the coefficient of $x_i$ in the original objective function while the coefficient for $Q_{ij} = Q_{ji}$ where $i \neq j$ is equal to half the original coefficient of $x_ij$ to keep the matrix symmetric. This is a simple QUBO problem and can be solved by enumerating the possible inputs to obtain the optimal solution of $x_1 = 0, x_2 = 1, x_3 = 1$.

\subsection{Knapsack problem}
The knapsack problem is a classic combinatorial optimization problem where we have a knapsack with weight capacity $W > 0$ and $n$ items each having a weight $w_i > 0$, profit $c_i > 0$. The problem is to find the optimal set of items to place in the knapsack to maximize total profit while not exceeding the knapsack weight limit.

Here, we can see an example of how QUBO solvers can also be used to solve combinatorial problems with inequality constraints. Formally, we have

\begin{align}
\max &\sum_{i=1}^n c_i x_i \nonumber\\
\mathrm{st.} &\sum_{i=1}^n w_i x_i \leq W \\
x_i &\in \{0,1\}^n \nonumber
\end{align}

To include the inequality constraint into our QUBO formulation, we introduce slack variables, $y_i$'s, which turn the inequality constraint into an equality constraint \cite{b6}.

\begin{align}
\max &\sum_{i=1}^n c_i x_i \nonumber\\
\mathrm{st.} &\sum_{i=1}^n w_i x_i = \sum_{j=1}^W jy_j\\
&\sum_{j=1}^W y_j = 1 \nonumber\\
x &\in \{0,1\}^n, y \in \{0,1\}^W \nonumber
\end{align}

With any penalty parameters $C_1 > \sum_{i=1}^n c_i$ and $C_2 > \sum_{i=0}^n c_i$, we can formulate the following QUBO problem 

\begin{align}
    \max f(x, y) &\coloneqq \sum_{i=1}^n c_i x_i - C_1 P_w^2 - C_2 P_n^2 \\
    P_w &\coloneqq \sum_{i=1}^n w_i x_i - \sum_{j=1}^W jy_j \\
    P_n &\coloneqq 1 - \sum_{j=1}^W y_j \\
    x &\in \{0,1\}^n, y \in \{0,1\}^W \nonumber
\end{align}

Since the penalty parameters are chosen to be larger than $\sum_{i=1}^n c_i$, the optimal solution to the QUBO problem must have $P_w = P_n = 0$. Hence, ensuring that exactly one of the $y_i=1$ and the total weight is below the knapsack capacity. Since the optimal solution to the original knapsack must also have a total weight below the knapsack capacity, the value of $x$ that solves the QUBO must also solve the original knapsack problem.

\subsection{Practical applications}
There are many practical scenarios where QUBO formulations can be utilized. 
\begin{itemize}
    \item \cite{b7} uses real-world data of the location of DB Schenker shipping hubs in Europe to solve the shipment rerouting problem which aims to reduce the total distance traveled to fulfill a set of shipments.
    \item \cite{b8} uses a QUBO formulation to solve portfolio optimization problems using real-world stock data sets of the New York Stock Exchange.
    \item \cite{b9} formulates an image binarization method as a QUBO problem where the objective is to segment an image into its foreground and background which has further possible medical applications to improve x-ray imaging.
\end{itemize}

\section{Ising model}
The Ising model in Physics can be thought of as a model of a magnet \cite{b11}. In the general Ising model, a magnet consists of $n$ molecules that are constrained to lie on the sites of a regular lattice \cite{b11}. Each molecule $i$ can be treated as a microscopic magnet that points along some axis and can have a 'spin' ($s_i)$ that is either $+1$ (parallel to the axis) or $-1$ (anti-parallel to the axis). With $N$ particles, the system can then have $2^N$ states, each corresponding to a configuration of the individual molecule spins.

\subsection{Ising Hamiltonian}
In quantum mechanics, the Hamiltonian is an operator that represents the total energy of a system. In the Ising model, the possible energies of the system are exactly the eigenvalues of its Hamiltonian and the corresponding eigenvectors are the possible states that result in the energy level \cite{b21}. The Hamiltonian of the Ising model has two components --- the external field term (characterized by $h$) and the interaction term between molecules (characterized by $J$) \cite{b10}. 

\begin{equation}
    H(s) = H(s_1, ... , s_n) = -\sum_{i < j} J_{ij}s_i s_j - \sum_{i=1}^N h_i s_i
\end{equation}

To find the ground state (lowest energy state) of the Ising model, we then have to solve for $\argmin H(s)$ for $s \in \{0, 1\}^N$. This is equivalent to a corresponding QUBO problem up to a change in basis. The equivalence of the QUBO problem and the Ising model is one of the most significant applications of QUBO \cite{b5}.

\subsection{Converting Ising to QUBO}
Given the Hamiltonian of an Ising problem, we use the conversion $s_i = 2x_i - 1, x_i \in \{0, 1\}$,

\begin{align}
    H(s) &= -\sum_{i < j} J_{ij}s_i s_j - \sum_{i=1}^N h_i s_i \nonumber\\
    &= -\sum_{i < j} J_{ij}(2x_i - 1) (2x_j - 1) - \sum_{i=1}^N h_i (2x_i - 1) \nonumber\\
    &= -\sum_{i < j} J_{ij}(4x_i x_j - 2x_i - 2x_j + 1) - \sum_{i=1}^N (2h_i x_i - h_i) \nonumber\\
    &\text{Group constants into $k$} \nonumber \\
    &\text{Let $a_i = \sum_{j \neq i} 2J_{\min(i,j)\max(i,j)} + 2h_i$} \nonumber \\
    &= -\sum_{i < j} 4J_{ij}x_i x_j - \sum_{i=0}^N a_{i}x_i + k \nonumber
\end{align}

Removing the constant $k$ which is irrelevant for optimization, we can reformulate the Ising model as a QUBO model with matrix $Q$ such that $Q_{ii} = -a_i$ and $Q_{ij} = Q_{ji} = -2J_{ij}$ for $i \neq j$. The solution to the QUBO problem can then be converted to a solution for the ground state of the Ising model using $s_i = 2x_i - 1$. However, note that the optimal objective function value may be different due to the constant $k$.

\subsection{Converting QUBO to Ising}

Given a QUBO problem with matrix $Q$, we can use the conversion, $x_i = \frac{s_i + 1}{2}, s_i \in \{-1, 1\}$. The objective function $f(x)$ of the QUBO problem can be expressed as

\begin{align}
    f(x) &= f(x_1, ..., x_n) \nonumber\\
    &= \sum_{i < j} Q_{ij}(x_i x_j) + \sum_{i=1}^N Q_{ii} x_i \nonumber \\
    &= \sum_{i < j} \frac{1}{4} Q_{ij}(s_i + 1)(s_j + 1) + \sum_{i=1}^N \frac{1}{2} Q_{ii} (s_i + 1) \nonumber \\
    &\text{Group constants into $k$} \nonumber \\
    &\text{Let $a_i = \sum_{j \neq i} \frac{1}{4}Q_{\min(i,j)\max(i,j)} + \frac{1}{2}Q_{ii}$} \nonumber \\
    &= \sum_{i < j} \frac{1}{4} Q_{ij}s_i s_j + \sum_{i=1}^N a_i s_i + k \nonumber \\
\end{align}

Removing the constant $k$ which is irrelevant for optimization, we can reformulate the QUBO problem as an Ising model with $h_i = -a_i$ and $J_{ij} = -\frac{1}{4}Q_{ij}$ for $i \neq j$. Similarly, the ground state for the Ising model can be converted to a solution for the QUBO problem and the optimal objective function value may be different due to the constant $k$.
\chapter{Methods for solving QUBO problems}

\label{review}
\vspace{2em}
There are a variety of possible methods for solving QUBO problems and Ising models which can be broadly categorized as Classical, Quantum Annealing, Neural Network Quantum States, and Hybrid Quantum-Classical Algorithms.

\section{Classical}
Classical approaches search for solutions without exploiting quantum properties such as the superposition of states. A typical classical approach to solving large QUBO problems is by exact diagonalization of the corresponding Ising Hamiltonian~\cite{b25}. Exact diagonalization solves for all the eigenvalues and eigenvectors by diagonalizing the corresponding Ising Hamiltonian matrix. This is also known as the eigendecomposition of a matrix and is possible for all Hermitian matrices, which are those that represent the Hamiltonian of a quantum system~\cite{b27}. However, the runtime of exact diagonalization scales exponentially with input problem size and becomes computationally infeasible once the matrix grows large \cite{b25}. Since we are only interested in finding the smallest eigenvalue and the corresponding eigenvector, we can use iterative methods such as the Lanczos algorithm or the implicitly restarted Arnoldi method to find the smallest eigenvalue~\cite{b28,b29}. However, such methods also often run into stability issues. The rapidly increasing search space for QUBO problems has inspired classical methods that aim to efficiently find approximate solutions to QUBO problems instead.

One class of such methods relies on "heuristics" to search for optimal solutions~\cite{b12}. For example, variants of Tabu search---a local search algorithm that allows for moves that are not improvements and discourages visiting already visited states---are highly competitive heuristic algorithms to find good solutions to QUBO problems\cite{b2,b30}. Dunning et al.\yrcite{b12} conducted a systematic review and evaluation of published heuristics for QUBO problems and provided an open-source problem repository MQLib for further research. 
%In addition, the solver by Dunning et al.\cite{b12} also provides a machine-learning model that predicts the best-performing set of heuristics for a given problem input.

Another classical approximate method for solving QUBO problems is simulated annealing (SA) \cite{KATAYAMA2001103}. Simulated annealing \cite{Kirkpatrick} is a probabilistic method for approximating the global minimum of a function. The main idea of SA is to start with an initial trial state and a temperature $T$ which decreases slowly during the search. In each iteration, the algorithm samples neighboring solutions of the current state and probabilistically accepts the new state based on the difference in energy, $\Delta E$, between the current state and the new state. If $\Delta E < 0$, then the new state is always accepted and if $\Delta E \geq 0$, the new solution is accepted with probability $\propto e^{-\frac{\Delta E}{T}}$, adapted from the Metropolis-Hastings sampling method \cite{metropolissampling}. The chance to explore poorer solutions allows for the algorithm to escape local minima.

There are many more classical approaches to solving QUBO problems as it has been studied extensively. For a detailed survey of classical methods for solving QUBO problems, refer to \cite{punnen2022quadratic}.

\section{Quantum Annealing}\label{section:annealing}
Quantum annealing can solve for the ground state configuration of Ising models by utilizing the \textit{adiabatic theorem} \cite{b14}. The \textit{adiabatic theorem} states that "a quantum system in its ground state will remain in the ground state, provided the Hamiltonian governing the dynamics changes sufficiently slowly" \cite{b14,b15}.

\begin{figure}[h!]
    \centering
    \includegraphics[width=0.8\linewidth]{images/quantum_annealing.png}
    \caption{Illustration of the energy landscape during quantum annealing}
    \label{quantumannealing}
\end{figure}

Quantum annealers first prepare a system in the ground state with a simple initial Hamiltonian $H_0$, then slowly manipulate the system Hamiltonian to a more complex form $H_c$ \cite{b10}. The Hamiltonian at any point $H(s)$, where $s \in [0,1]$ is the normalized anneal fraction, can be written as

\begin{equation}
    \label{eqn:annealinghamiltonian}
    H(s) = A(s)H_0 + B(s)H_c
\end{equation}

Where $A(s)$ and $B(S)$ are decreasing and increasing functions respectively and are determined by the quantum annealing controls. If the transition time is sufficiently large, and $H_0$ and $H_c$ do not commute, the adiabatic theorem ensures that the system will remain in the ground state which can then be measured to yield the desired ground state configuration \cite{b14}.

Quantum annealing has been extensively studied and applied in multiple fields such as Scheduling Problems \cite{b17}, Portfolio Optimization \cite{b18} and Quantum simulations \cite{b19}. Even though it is debated whether Quantum Annealing will be superior to classical methods \cite{b10} and there are significant roadblocks in scaling hardware capabilities \cite{b14}, there is hope that these challenges can be tackled soon with the rapid progress made in quantum computing research. The current leading commercial provider of quantum annealing hardware is D-wave, a Canadian quantum computing company \cite{b16}.

\section{Neural-Network Quantum States}
Carleo and Troyer \cite{b20} introduced another quantum-based method for modeling the wave function of a given Ising Hamiltonian known as \textit{neural-network quantum states} (NNQS). The wave function can be viewed as a complex probability distribution over all the possible state configurations and the NNQS method approximates the wave function of a system as an artificial neural network \cite{b25}. 

The original NNQS architecture utilized a Restricted Boltzmann Machine (RBM) to represent the wave function of an arbitrary quantum system \cite{b20}. The RBM is an energy-based machine learning model that has two layers of nodes, a visible layer $\boldsymbol{s}$, and a hidden layer $\boldsymbol{h}$, where each visible node $s_i$ represents the spin of a particle in the Ising model \cite{b20}. Each visible node $s_i$ is connected to every hidden node $h_j$ with a certain weight $W_{ij}$ but there are no connections within the visible or hidden layer \cite{b20}. The representation of the wave function by the RBM can then be expressed as

\begin{equation}
    \Psi(\boldsymbol{s}, \boldsymbol{\theta}_{rbm}) = \sum_{h} e^{\sum_i a_s s_i + \sum_j b_j h_j + \sum_{i,j}W_i s_i h_j} 
\end{equation}

where  $\boldsymbol{\theta}_{rbm} = \{\boldsymbol{a}, \boldsymbol{b}, \boldsymbol{W}\}$ are the biases and weights of the RBM \cite{b20}. The RBM is trained in an unsupervised manner. The weights $W$ are usually updated by performing Gibbs sampling, calculating the average energy of sampled configurations, and using gradient-based optimization algorithms \cite{b25}.

Other neural network architectures such as the Multilayer Perceptron (MLP) can also be used for NNQS. An MLP model consists of an input layer $\boldsymbol{x}$, one or more hidden layers, and an output layer. Each layer is fully connected to the next layer with certain weights and each node has an activation function $\sigma$ such as the sigmoid or ReLU. Each input node $x_i$ represents the spin of a particle in the Ising model and the output nodes represent the output from the wave function. If we assume that the wave function is real and positive, then only one output node is needed to represent the real part of the wave function. \cite{b20}. With one hidden layer, the MLP representation of the wave function can be expressed as

\begin{equation}
    \Psi(\boldsymbol{x}, \boldsymbol{\theta}_{mlp}) = 
    \sigma_{out} \left(
    \boldsymbol{W}_{out}
    \sigma_{hidden} \left( \boldsymbol{W}_{hidden}\boldsymbol{x} + \boldsymbol{b}_{input} \right) + \boldsymbol{b}_{hidden} \right)
\end{equation}

where  $\boldsymbol{\theta}_{mlp} = \{\boldsymbol{W}_{hidden}, \boldsymbol{b}_{input}, \boldsymbol{W}_{out}, \boldsymbol{b}_{hidden}\}$ are the weights and bias of the MLP \cite{b20}. The MLP is trained in an unsupervised manner, similar to the RBM, except that a more general sampling method, the Metropolis algorithm, has to be used \cite{b25}.

\section{Hybrid quantum-classical}
The fourth class of QUBO solving methods are hybrid quantum-classical methods which are designed to make use of current limited quantum computing resources by integrating them with classic methods to solve combinatorial optimization problems \cite{b32}. In the noisy intermediate-scale quantum (NISQ) era that we are currently in, quantum computers are not yet stable enough to be able to  One such algorithm is the Quantum Approximate Optimization Algorithm (QAOA) which is used to find approximate solutions to general combinatorial optimization problems \cite{b23}. Similar to NNQS, QAOA aims to find an approximation of the ground state of an input Hamiltonian using gate-based quantum circuits and $2p$ parameters which are optimized with a classical computer \cite{b34}. With a problem of size $N$. the algorithm first prepares a quantum state $| + \rangle^{\otimes N}$ in uniform superposition by applying the Hadamard gate to each of the $n$ inputs \cite{b34}. The trial wave function is then constructed by

\begin{equation}
    \Psi(\boldsymbol{\gamma}, \boldsymbol{\beta}) = U_B(\beta_p) U_C(\gamma_p)...U_B(\beta_1) U_C(\gamma_1) | + \rangle^{\otimes N}
\end{equation}
\begin{align*}
    U_C(\gamma) &= e^{-i\gamma H_c} \\
    U_B(\beta) &= e^{-i\beta H_0}
\end{align*}

$U_c$ and $U_B$ are operators that evolve the state with the Hamiltonian $H_c$ (problem Hamiltonian) and $H_0$ (an easy-to-implement Hamiltonian) for times $\boldsymbol{\gamma}$ and $\boldsymbol{\beta}$ and the parameter $p$ determines the number of independent parameters of the final state \cite{b34}. The state is then measured and $\boldsymbol{\gamma}$ and $\boldsymbol{\beta}$ are chosen to minimize the average energy of the sample measurements. With the optimal parameters, the final solution can then be determined by repeatedly sampling from the trial wave function and using the state that occurs most frequently during measurements.

In contrast to quantum annealing which can only be run on specialized quantum annealing devices, QAOA can be implemented on a general gate-based quantum computer \cite{b22}. 
\SetPicSubDir{ch-methodology}
\SetExpSubDir{ch-methodology}

\chapter{Methodology}
\label{methodology}
This chapter explains the methodology employed in our study, which includes the solvers, datasets, and evaluation metrics.

\section{Solvers}
We will employ five QUBO solvers for the study:
\begin{enumerate}
    \item D-Wave Quantum Annealing
    \item Neural Network Quantum States (NNQS)
    \item Quantum Approximate Optimization Algorithm (QAOA)
    \item GUROBI Optimizer
    \item Fixstars Amplify QUBO Solver
\end{enumerate}
The following sections will provide more information on the solvers.

\subsection{D-Wave Quantum Annealing}
We will use quantum annealers from D-Wave Systems, which produce the most popular commercially available quantum annealing hardware. The annealers are accessed with D-Wave's Solver API and use radio frequency superconducting quantum–interference device (rf-SQUID) qubits to represent spin variables during the annealing process \cite{b14}. External magnetic fields represent the linear interactions ($h$ terms), while couplers help entangle pairs of qubits to represent the interaction between qubits ($J$ terms). The qubits are arranged in the Pegasus graph topology shown in \autoref{pegasustopology} that allows up to 15 couplers per qubit \cite{b14}. 

\begin{figure}[htp]
    \centering
    \includegraphics[width=0.4\linewidth]{images/pegasus_topology.png}
    \caption[A view of the D-Wave pegasus topology. Each line represents a qubit, and intersections represent available couplers. Each qubit can only be coupled to at most $15$ other qubits.]{A view of the D-Wave pegasus topology. Each line represents a qubit, and intersections represent available couplers. Each qubit can only be coupled to at most $15$ other qubits. ~\protect\cite{dwaveadvantage}}
    \label{pegasustopology}
\end{figure}

The experiments will be conducted with the D-Wave Advantage 4.1 QPU, which has up to 5640 qubits and 40484 couplers \cite{dwaveadvantage}. Given a target QUBO problem to solve, the process for using a D-Wave annealer is as follows:
\begin{enumerate}
    \item \textbf{Problem Definition} QUBO problem is first converted to its corresponding Ising model.
    \item \textbf{Minor Embedding} As the D-Wave quantum processing unit shown in \autoref{pegasustopology} is not fully connected, a single spin variable may need to be represented by multiple qubits called a \textit{chain}, which are forced to return the same value with large interaction terms \cite{b16}. The EmbeddingComposite class in the D-Wave library performs the embedding.
    \item \textbf{Programming} The parameters of the annealing process are set, which consists of the linear term for each qubit (with an external magnetic field acting on each qubit) and coupler strength (represents variable interaction between spins).
    \item \textbf{Initialization} The QPU is initialised in the ground state of the initial Hamiltonian, which is usually a superposition of all possible states.
    \item \textbf{Annealing} The system evolves with a time-varying Hamiltonian:
    \begin{equation}
        \label{eqn:dwavehamiltonian}
        H(s) = -\frac{A(s)}{2}\left(\sum_{i} \Hat{\sigma}_x^{(i)}\right) + \frac{B(s)}{2}\left(\sum_{i}h_i \Hat{\sigma}_z^{(i)}\ + \sum_{i > j}J_{i,j} \Hat{\sigma}_z^{(i)}\Hat{\sigma}_z^{j)}\right)
    \end{equation}
    where $s$ is the normalised anneal fraction and $A(s), B(s)$ are annealing functions shown in \autoref{dwaveannealing}. We will use the default annealing time of $20\mu s$. 
    \item \textbf{Readout of solution} The spin values of the qubits are measured and stored as a possible solution.
    \item \textbf{Resample} As finite-time quantum annealing does not guarantee optimality, we repeat the annealing and sampling process for $1000$ iterations and use the sample with the lowest energy as the candidate solution.
\end{enumerate}

\begin{figure}[h!]
    \centering
    \includegraphics[width=0.5\linewidth]{images/dwave_annealing.png}
    \caption[Annealing functions $A(s)$ and $B(s)$ as a function of the normalised anneal fraction $s$. $A(s) >> B(s)$ when $s=0$ and $B(s) >> A(s)$ when $s=1$. The functions have an energy scale of around $10^{-24} J$.]{Annealing functions $A(s)$ and $B(s)$ as a function of the normalised anneal fraction $s$. $A(s) >> B(s)$ when $s=0$ and $B(s) >> A(s)$ when $s=1$. The functions have an energy scale of around $10^{-24} J$. ~\protect\cite{dwaveadvantage}}
    \label{dwaveannealing}
\end{figure}

\subsection{Neural-Network Quantum States}
We will adopt the Python library MAPALUS for implementing Neural-Network Quantum States \cite{b25}. MAPALUS uses the Tensorflow library and enables parallel execution on general-purpose graphics processing units for quicker sampling. We will use the Restricted Boltzmann Machine with $5n$ hidden nodes and the sigmoid activation function as the underlying architecture for the NNQS.

To solve a QUBO problem, the NNQS simulates the D-Wave quantum annealing process on a classical computer with a time-dependent Hamiltonian that follows \autoref{eqn:annealinghamiltonian}. Since the equations for $A(s)$ and $B(s)$ are unavailable, we employ a curve-fitting process to obtain analytical functions from the discrete points provided by D-Wave. The annealing functions used are $A(s) = 1.11e^{-7.06s} + -0.00569$ and $B(s)= 0.680s^2 + 0.288s + 0.0305$. The curve fitting process is detailed in \autoref{appendix:curvefitting}.

The NNQS will be trained with a progressive training schedule that mimics the quantum annealing process and follows \autoref{alg:progressive}. $\hat{H}_c$ is the problem Hamiltonian and $\hat{H}_0$ is a Hamiltonian with linear biases as $1$ and no quadratic terms. The normalised anneal fraction $s$ is increased in small steps while the NNQS is trained to convergence, which simulates the slow change in the system Hamiltonian during quantum annealing. Like quantum annealing, the NNQS remains in the ground state throughout the training as it is trained to convergence.

\begin{algorithm}
    \begin{algorithmic}
    \Require Problem Hamiltonian $\hat{H}_c$
    \Ensure Trained NNQS
    \State Initialize NNQS with random weights;
    \For {$s \in [0.1, 1.0]$ step $0.1$}
    \State Set $H(s) \leftarrow A(s)\hat{H}_0 + B(s)\hat{H}_c$;
    \State Train NNQS on $H(s)$ until convergence or until epoch limit of $100$ is reached;
    \EndFor
    \end{algorithmic}
    \caption{NNQS Progressive Training}
    \label{alg:progressive}
\end{algorithm}

The training process of the NNQS involves updating the parameters, $\boldsymbol{\theta}$, with a Variational Monte Carlo approach to minimise the energy expectation value. The training algorithm is described as follows:
\begin{enumerate}
    \item Sample a set of $1000$ configurations $\mathbf{s}$ from the probability distribution defined by the NNQS, $\probP(s) = |\Psi(s;\theta)|^2$, with Gibbs sampling \autoref{samplingmethods}.
    \item Calculate an unbiased estimate of the energy expectation value with the average energy of the sampled configurations.
    \item Compute the gradients of the energy expectation value with respect to the NNQS parameters using backpropagation.
    \item Update the parameters with Root Mean Squared Propagation (RMSprop), a gradient-based optimiser, with a learning rate of $0.001$ \cite{rmsprop}.
    \item Repeat steps 1-4 for $1000$ iterations or until convergence is reached.
\end{enumerate}
The derivation of the gradients is detailed in the appendix of \cite{b20}. The candidate solution will be the best among the $1000$ configurations sampled from the final NNQS. All NNQS experiments were run on a 32 Core AMD 7543P Processor and an NVIDIA A100 40GB GPU with 500GB of RAM.

\subsection{Quantum Approximate Optimization Algorithm}
We will implement QAOA in Qiskit, an open-source software development kit for quantum computing, with $p=1$ using a backend hosted on the IBM Quantum Platform (IBMQ) \cite{b24}. We will utilise the IBMQ simulator, \texttt{ibmq\_qasm\_simulator}, a general-purpose simulator for quantum circuits that can handle up to 32 qubits. Even though IBMQ allows for access to quantum computers, it is limited by long wait times ($>5$ hours for each problem) and is infeasible for a benchmarking experiment. To measure the optimal performance of the QAOA algorithm, we will assume an ideal circuit without a quantum noise model. We will use the default mixing Hamiltonian of $\Hat{H}_0 = \sum_{i}\Hat{\sigma}_x^{(i)}$ and the Qiskit transpiler to optimise the decomposition of the operators into their individual parametrised quantum gates as shown in \autoref{qiskitcircuit} \cite{qiskittranspiler}.

\begin{figure}[htp]
    \centering
    \includegraphics[width=1\linewidth]{images/qiskit_circuit.png}
    \caption{Circuit diagram with the Hadamard gates, problem Hamiltonian operator, mixing Hamiltonian operator and measurement gates (left to right)}
    \label{qiskitcircuit}
\end{figure}

The parameters $(\boldsymbol{\gamma}, \boldsymbol{\beta})$ are updated to minimise the energy expectation value. The training algorithm is described as follows:
\begin{enumerate}
    \item Initialise the circuit in the initial state $| + \rangle^{\otimes n}$ where each qubit is in a superposition state. Initialise $\gamma$ and $\beta$ with values sampled from a standard Gaussian normal distribution. 
    \item Construct the operators $U_C(\gamma), U_B(\beta)$ with $\gamma, \beta$, which consists of different quantum gates.
    \item Repeatedly sample the final states of the qubits and calculate an unbiased estimate of the energy expectation value as the average energy of the sampled configurations.
    \item Use a classical derivative-free optimiser Constrained Optimisation by Linear Approximation (COBYLA) to optimise the parameters $\gamma, \beta$ to minimise the energy expectation value.
    \item Repeat steps 2-4 for some iterations or until convergence is reached.
\end{enumerate}
We will repeatedly sample the final optimised circuit $1000$ times, and the candidate solution will be chosen as the best solution among the sampled configurations. The QAOA solver is accessed with the IBM Quantum Qiskit Runtime API and is run on the IBMQ cloud servers.

\subsection{GUROBI Optimizer}
The GUROBI optimiser is a state-of-the-art classical commercial solver \cite{b26}. As the GUROBI optimiser supports QUBO problems, we construct the QUBO matrix directly using the Gurobi Python library and run the optimiser locally for $10$ minutes for each input problem. If the optimiser finishes before the cutoff, the candidate solution is guaranteed to be the optimal configuration. Otherwise, we would use the best solution within the cutoff time as the candidate solution. The GUROBI optimiser uses a branch and bound algorithm that first relaxes the integer constraint of the QUBO problem, then branches into subproblems based on variables that violate the constraints. All experiments were run on a 32 Core AMD 7543P Processor using Gurobi Optimizer version 10.0.3.

%All GUROBI experiments used the Gurobi Optimizer version 9.0.1 and were run on a local machine with an 8-core Apple M1 chip at 3.2GHz with 16GB of RAM.

\subsection{Fixstars Amplify QUBO Solver}
The Fixstar Amplify QUBO solver is a commercial simulated annealing-based QUBO solver \cite{b12}. As the Fixstar solver supports QUBO problems, we submit the QUBO matrix directly using the Fixstar API and run the solver with the highest allowed time limit of $100$ seconds for each input problem. The solver is implemented on GPUs and runs on Fixstar remote servers.

\section{Benchmark datasets}
We use three randomly generated problem sets to benchmark our QUBO solvers: not-all-equal 3-satisfiability (NAE3SAT), max-cut, and the Sherrington-Kirkpatrick (SK) model. These problems were chosen as they are commonly used in QUBO experiments to represent NP-hard problems and are relatively straightforward to encode into QUBO form. The NAE3SAT and max-cut problem sets are macro benchmarks (application-based) to represent practical combinatorial optimisation problems. In contrast, the SK model problem set is a microbenchmark designed as a difficult problem set. Each problem set comprises problems of 13 sizes, ranging from $10$ to $300$\footnote{$n \in [10,15,20,25,30,35,50,75,100,150,200,250,300]$}. $20$ different random problems are generated\footnote{random seed is chosen to be from $0-19$} for each problem size for a total of $260$ problems per problem set. Each problem is first formulated in either the QUBO or Ising form, and the conversion between them follows \autoref{qubotoising} and \autoref{isingtoqubo}.

\subsection*{Not-all-equal 3-satisfiability (NAE3SAT)}
The NAE3SAT problem is a variant of the boolean satisfiability problem where each problem instance consists of $n$ boolean variables $(x_1, x_2, ..., x_n)$ and $m$ clauses that each combine three literals, which can be a variable or its negation. The objective is to find an assignment of boolean values such that the three values in each clause are not all the same, i.e., each clause has at least one true and one false value. We generate random NAE3SAT problems with $\rho = \frac{m}{n} = 2.1$, where NAE3SAT problems are known to transition from being satisfiable to unsatisfiable \cite{nae3sattransition}, using the random\_nae3sat generator from the dimod Python library, which uniformly samples 3-variable clauses with replacement \cite{dimodrandomnae3sat}. 

To convert a NAE3SAT problem into an Ising problem, we represent each boolean variable as a spin variable and turn each clause into a Hamiltonian term. For example, the clause $(x_1, x_2, \neg x_3)$, becomes the Hamiltonian term $H(s_1, s_2, s_3) = s_1 \cdot s_2 + s_2 \cdot (-s_3) + s_1 \cdot (-s_3)$ which has a value of $3$ when $x_1=x_2=\neg x_3$ and $-1$ otherwise. The final Hamiltonian $\Hat{H}_c$ is simply a sum of the individual Hamiltonian terms for each clause.

\subsection*{Max-cut}
The max-cut problem aims to find a partition of the vertices of a graph $G = (V, E)$ into $V_0, V_1$ with $V = V_0 \cup V_1$ and $V_0 \cap V_1 = \emptyset$, to maximise the number of edges crossing $V_0$ and $V_1$. We will use the Erdos-Renyi model to generate random graphs with $n$ vertices and a probability $p=0.25$ for each edge to be in $E$.

To convert a max-cut problem into a QUBO problem, we represent the assignment of each vertex as a binary variable ($x_v = i$ if $x \in V_i$) and use the objective function $f(\mathbf{x}) = \sum_{e = (u, v) \in E} -x_u - x_v + 2x_u x_v$, where each term has a value of $-1$ when $x_u  \neq x_v$ and $0$ otherwise. When the cut value is maximised, $f(\mathbf{x})$ is minimised. 

\subsection*{Sherrington-Kirkpatrick (SK) model}
The Sherrington Kirkpatrick (SK) model is an Ising problem with a random Hamiltonian of the form $\hat{H}_c = \frac{1}{\sqrt{n}} \sum_{1 \leq i < j \leq n} J_{ij}s_i s_j$
where $J_{ij} \sim \mathcal{N}(0,1)$ are independent standard Gaussian variables. The energy landscape of the SK model is complex as it has exponentially many local minima with a unique global minimum separated by high energy barriers \cite{skmodel}. This many-valley structure implies that finding the exact solution of the model is challenging. We generate random Gaussian variables with a random normal generator from the NumPy Python library.

%Parisi [6] provides a formula (for proof, see [7]) that, when numerically evaluated [8, 9, 1], shows for typical instances,
%\begin{equation*}
%    \lim_{n\rightarrow \infty} \argmin \frac{E}{n} = -0.763166...
%\end{equation*}
%(get from this paper https://quantum-journal.org/papers/q-2022-07-07-759/pdf/)

%\subsection*{Quantum Critical Points}
%get rid of linear term
%J terms are 1
%Ising model has no linear term
%QCP is A(s)/2 = +-B(s)/2 * J magnitude is the same.
%harder for Dwave to solve, jump to excited state
%finite infinite approximating

%We plan to use a subset of the 3296 QUBO problems provided by MQLib as our benchmark dataset. These problems are publicly accessible and contain both "real-world problem instances" and randomly generated problems \cite{b12}. The dataset also contains problems of various sizes and densities. The exact benchmark dataset will be determined after further testing to determine the input limits of each solver. Each QUBO problem will first be converted into an Ising model problem and then passed to each solver.

\section{Performance evaluation}
We use two metrics to evaluate the performance of the solvers, which are calculated separately for each problem type and size:
\begin{enumerate}
    \item The success probability, which is the probability of finding a solution with the lowest energy among all solutions found by the $5$ solvers. Since we are generating $20$ problems of each type and size, the success probability for each solver is:
    \begin{equation}
        \Bar{p} = \frac{\text{number of lowest energy solutions found}}{20}
    \end{equation}
    \item The average normalized energy, used in \cite{b34}, which is:
    \begin{equation}
        \Bar{r}_{solver} =  \frac{1}{20} \sum_{i = 1}^{20} \frac{E^i_{max} - E^i_{solver}}{E^i_{max} - E^i_{min}}
    \end{equation}
    where $E^i_{max}$ and $E^i_{min}$ are the energies of the worst and best solutions found by all solvers for problem $i$, and $E^i_{solver}$ is the energy that the solver found for problem $i$. Note that if a solver finds the best solution, it gets a normalised energy of $1$; if it finds the worst solution, it gets a normalised energy of $0$. All solvers get a normalised energy of $1$ if all solutions have the same energy.
\end{enumerate}
The average normalised energy measures the quality of the solutions produced by the solvers, while the success probability measures the solvers' ability to find the best solution.
\chapter{Benchmarking QUBO solvers}\label{benchmark}
This chapter first introduces related benchmarking work for QUBO problems and QUBO solvers. Then, we present our results for the solvers and datasets used for this study.

\section{Related benchmarking work}
One of the first benchmarking works for quantum annealing was conducted by Denchev et al. \yrcite{denchev2016computational}, who measured the performance of D-Wave quantum annealing on the older D-Wave 2X machine using specially crafted problems that have tall and narrow energy barriers separating local minima. Quantum annealing is expected to be $1.8 \times 10^8$ times faster than simulated annealing, which tends to fail with problems with such an energy landscape.


\outcite{b34} evaluated the performance of QAOA on the IBMQ backend and the D-Wave solver using instances of MaxCut and 2-satisfiability problems with up to 18 variables. The performance of the QAOA algorithm is inconsistent and underperforms quantum annealing in their problem set. More recently, \outcite{b35} also compared the performance of QAOA on the IBMQ backend and D-Wave quantum annealing on randomly generated Ising problems with cubic interaction terms and found that quantum annealing had superior performance over QAOA for all problem sizes.

\outcite{gomes2019classical} showed that the NNQS solving method with an RBM architecture produces good quality solutions for the max-cut problem with graph sizes of up to 256. \outcite{khandoker2023supplementing} uses recurrent neural networks as the NNQS architecture for the max-cut and travelling salesman problem and found that it outperforms SA. However, no direct study has compared performance across quantum annealing, QAOA, and NNQS.

\section{Results and Discussion}
Performance is shown for each dataset, accompanied by error bars representing each data point's unbiased standard error of the mean. Graphs with problem sizes on the x-axis are plotted with a log scale.

\subsection{NAE3SAT}

\begin{figure}[!htbp]
    \centering
    \subfloat[Normalized energy]{\includegraphics[width=0.9\textwidth]{images/nae3sat_all_size.png}}
    \\
    \subfloat[Success probability]{\includegraphics[width=0.9\textwidth]{images/nae3sat_all_success_size.png}}
    \caption{Performance of different solvers for NAE3SAT by problem size}
    \label{all-nae3sat-size}
\end{figure}

\begin{figure}[!htbp]
    \centering
    \subfloat[Normalized energy]{\includegraphics[width=0.49\textwidth]{images/nae3sat_all_avg.png}}\hfill
    \subfloat[Success probability]{\includegraphics[width=0.49\textwidth]{images/nae3sat_all_success_avg.png}}
    \caption{Average performance of different solvers for NAE3SAT}
    \label{all-nae3sat-average}
\end{figure}

Performance by size for the NAE3SAT dataset in \autoref{all-nae3sat-size} and average performance is shown in \autoref{all-nae3sat-average}. The D-wave solver and NNQS could both solve problems up to $n=300$. However, multiple embedding requests were required for problems of size $300$ for the D-wave Pegasus topology, which suggests that $n=300$ might be near the D-wave size limit for the NAE3SAT problem. QAOA could only solve problems of up to $n=30$ due to the limitations on the number of qubits of the simulator.

In terms of performance, the D-wave solver performs well up to $n=50$ with a success probability of $1$. For larger problem sizes, the performance of the D-wave solver drops off sharply. The NNQS performs well up to $n=20$, with the success probability and normalised energy gradually decreasing until $n=300$. The QAOA solver performs well up to $n=15$, and performance decreases until $n=30$. Between the classical solvers, the Fixstars QUBO solver performs better than the GUROBI optimiser at larger problem sizes ($>150$). Both classical solvers outperform the quantum-inspired solvers.

Overall, the NNQS has the highest average normalised energy among the three quantum-inspired solvers but has the lowest success probability, which is likely due to it being able to solve problems of larger sizes that the D-wave Annealer and QAOA solver cannot handle. The QAOA solver has the highest success probability but can only handle problems up to $n=30$.

\subsection{Max-cut}

\begin{figure}[!htbp]
    \centering
    \subfloat[Normalized energy]{\includegraphics[width=0.9\textwidth]{images/maxcut_all_size.png}}
    \\
    \subfloat[Success probability]{\includegraphics[width=0.9\textwidth]{images/maxcut_all_success_size.png}}
    \caption{Performance of different solvers for max-cut by problem size}
    \label{all-maxcut-size}
\end{figure}

\begin{figure}[!htbp]
    \centering
    \subfloat[Normalized energy]{\includegraphics[width=0.49\textwidth]{images/maxcut_all_avg.png}}\hfill
    \subfloat[Success probability]{\includegraphics[width=0.49\textwidth]{images/maxcut_all_success_avg.png}}
    \caption{Average performance of different solvers for max-cut}
    \label{all-maxcut-average}
\end{figure}

Performance by size for the max-cut dataset is shown in \autoref{all-maxcut-size}, and average performance is shown in \autoref{all-maxcut-average}. For the max-cut problem, the D-wave solver could only handle problem sizes up to $n=150$ due to the need for minor embedding onto the pegasus topology. QAOA solved problems of up to $n=30$ due to the limitations of the simulator.

In terms of performance, the D-wave solver performs well up to $n=30$, and performance drops off sharply for larger problems. The NNQS performs well at $n=150$, although the success probability decreases for problem sizes over $50$. The QAOA solver performs well only for $n=10$, with performance decreasing for larger problems.

Overall, the NNQS has the highest average normalised energy among the three quantum-inspired solvers and has a slightly lower success probability than the D-wave solver. The QAOA solver performs poorly in both metrics.

\subsection{SK model}

\begin{figure}[!htbp]
    \centering
    \subfloat[Normalized energy]{\includegraphics[width=0.9\textwidth]{images/skmodel_all_size.png}}%\hfill
    \\
    \subfloat[Success probability]{\includegraphics[width=0.9\textwidth]{images/skmodel_all_success_size.png}}
    \caption{Performance of different solvers for SK model by problem size}
    \label{all-skmodel-size}
\end{figure}

\begin{figure}[!htbp]
    \centering
    \subfloat[Normalized energy]{\includegraphics[width=0.49\textwidth]{images/skmodel_all_avg.png}}\hfill
    \subfloat[Success probability]{\includegraphics[width=0.49\textwidth]{images/skmodel_all_success_avg.png}}
    \caption{Average performance of different solvers for SK model}
    \label{all-skmodel-average}
\end{figure}

Performance by size for the SK model dataset is shown in \autoref{all-skmodel-size}, and average performance is shown in \autoref{all-skmodel-average}. For the SK model, the D-wave solver could only handle problem sizes up to $n=150$ due to the need for minor embedding onto the pegasus topology. The SK model is fully connected, which makes embedding difficult for the D-wave QPU. QAOA solved problems of up to $n=30$ due to the limitations of the simulator.

Due to its multi-valley energy landscape, the SK model presents a difficult problem for all QUBO solvers. The D-wave solver performs well up to $n=30$, gradually decreasing performance for larger problems. The NNQS follows a similar trend, although it can solve problems of larger sizes and performs better at sizes of $=100,150$. The QAOA solver has consistently poor performance across problem sizes.

Overall, the D-wave annealer has the highest average normalised energy and success probability among the three quantum-inspired solvers. The NNQS is slightly worse in both metrics, while the QAOA solver performs poorly for both metrics.

\section{Time-Constrained Solver Comparison}
We also measured the runtime for each solver for each problem and the average runtime across all problems with the same size $n$ shown in \autoref{results:timeaverage}. Runtimes split by problem type can be found in \autoref{appendix:timesizegraph}

For the D-wave solver, the average runtime increases approximately linearly from $0.128\si{\second}$ for $n=10$ to $0.184\si{\second}$ for $n=50$ and $0.292\si{\second}$ for $n=300$. 

For the NNQS solver, the runtime does not increase significantly from $n=10$ to $n=150$ and remains around from $250\si{\second}$ to $350\si{\second}$ but increases sharply for larger $n \geq 200$ which could be due to memory issues with the GPU or the sampling may have taken a longer time to converge. 

The QAOA solver's runtime remains stable from $n=10$ to $n=20$ at around $38 \si{\second}$. However, it increases rapidly to $6847 \si{\second}$ at $n=30$ due to the need for more optimisation iterations and greater computational resources for the simulation.

The GUROBI optimiser was set with a maximum time limit of $600 \si{\second}$ but could finish solving before the time limit for most problems with $n \leq 35$. The Fixstar solver was configured with a maximum time limit of $100 \si{\second}$, the maximum possible duration, and each solve utilised the entire time limit.

\begin{figure}[!htbp]
    \centering
    \includegraphics[width=0.9\textwidth]{images/all_time_average.png}
    \caption{Average runtime in log scale taken by different solvers for QUBO problems by size}
    \label{results:timeaverage}
\end{figure}

Using the average runtime of the D-wave solver, we conducted a second run of the benchmarking to measure the performance of the D-wave solver against the classical solvers---GUROBI and Fixstar. This experiment aimed to test if the D-wave solver could outperform the classical solvers if they were limited to the same runtime. For a problem of a specific type and size, the classical solvers were run with a maximum runtime equal to the average time required by the D-wave solver for problems of equivalent type and size shown in \autoref{results:averageruntimedwave}.

\begin{table}[!ht]
    \centering
    \resizebox{\textwidth}{!}{%
    \begin{tabular}{lrrrrrrrrrrrrr} \toprule
        $n$ & 10 & 15 & 20 & 25 & 30 & 35 & 50 & 75 & 100 & 150 & 200 & 250 & 300 \\ \midrule
        NAE3SAT & 0.133 & 0.135 & 0.141 & 0.149 & 0.138 & 0.156 & 0.173 & 0.184 & 0.201 & 0.243 & 0.259 & 0.286 & 0.292 \\
        Max-cut & 0.127 & 0.136 & 0.1560 & 0.130 & 0.155 & 0.160 & 0.183 & 0.223 & 0.247 & 0.278 & - & - & - \\
        SK model & 0.125 & 0.137 & 0.140 & 0.140 & 0.150 & 0.161 & 0.195 & 0.221 & 0.245 & 0.278 & - & - & - \\ \bottomrule
    \end{tabular}}
    \caption{Average runtime (seconds) of the D-wave solver by problem type and size. Dashes indicate that the D-wave solver could not embed problems of that size.}
    \label{results:averageruntimedwave}
\end{table}

The results are shown in \autoref{all-time-size}. For each problem type, the performance of the GUROBI solver drops before the D-wave solver, and there are problem types and sizes where the D-wave solver outperforms the GUROBI solver when the runtime is matched. The D-wave solver outperforms GUROBI for NAE3SAT with $n=50, 75$, max-cut with $n=30,35$, and SK model with $n=20, 35$. However, when the problem sizes increase even further, the D-wave solver performs poorly, possibly due to the increased noise of the quantum annealer. The Fixstar solver remains the top-performing solver across all problem types and sizes, even when the runtime is matched with the D-wave solver. The results show that the D-wave solver can outperform classical solvers like GUROBI for specific problem sizes when the runtime is matched.

\begin{figure}[!htbp]
    \centering
    \subfloat[NAE3SAT]{\includegraphics[width=0.7\textwidth]{images/nae3sat_timing_size.png}}%\hfill
    \\
    \subfloat[Max-cut]{\includegraphics[width=0.7\textwidth]{images/maxcut_timing_size.png}}
    \\    
    \subfloat[SK model]{\includegraphics[width=0.7\textwidth]{images/skmodel_timing_size.png}}
    \caption{Performance of D-wave solver against GUROBI and Fixstar by problem type and size}
    \label{all-time-size}
\end{figure}

\section{Solver logs}
During the benchmarking experiments, supplementary metadata from the D-wave and QAOA solver was recorded. We recorded the embedded qubit count for the D-wave solver, which tends to grow differently for different problem types. For the QAOA solver, we recorded the number of quantum gates and the circuit depth, which can help quantify the complexity of the quantum circuit used. A metadata summary is available in \autoref{appendix:metadata}.

\section{Conclusion}
When comparing the three quantum-inspired solvers, NNQS has the best normalised energy for the NAE3SAT and max-cut dataset, while the D-wave solver has the best normalised energy for the SK model performance. NNQS also does the best in normalised energy when averaged across the three datasets. NNQS tends to do better in normalised energy since it minimises the energy expectation value, which optimises the average energy of samples but does not necessarily optimise for the highest probability of sampling the best solution. 

QAOA has the highest success probability for the NAE3SAT dataset. However, it is important to note that it could only handle problems with up to $30$ variables. The D-wave solver has the best success probability for the max-cut and SK model datasets and the average success probability across the three datasets.

\autoref{results:allnormalizedenergy} and \autoref{results:allsuccess} show the average normalised energy and success probability for different solvers for each dataset and the average across all datasets. Across all datasets and both metrics, the Fixstar QUBO solver has the best performance, consistently returning the best solutions out of all solvers.


\begin{table}[!ht]
    \centering
    \begin{tabular}{cccccc} \toprule
        ~ & D-wave & NNQS & QAOA & GUROBI & Fixstar \\ \midrule
        NAE3SAT & 0.617 & \textbf{0.755} & 0.640 & 0.983 & 1.00 \\
        Max-cut & 0.610 & \textbf{0.753} & 0.190 & 0.991 & 1.00 \\
        SK model & \textbf{0.761} & 0.664 & 0.230 & 0.990 & 1.00 \\ \midrule
        Average & 0.663 & \textbf{0.724} & 0.353 & 0.988 & 1.00 \\ \bottomrule
    \end{tabular}
    \caption{Average normalised energy for different solvers}
    \label{results:allnormalizedenergy}
\end{table}

\begin{table}[!ht]
    \centering
    \begin{tabular}{cccccc} \toprule
        ~ & D-wave & NNQS & QAOA & GUROBI & Fixstar \\ \midrule
        NAE3SAT & 0.615 & 0.538 & \textbf{0.640} & 0.858 & 1.00 \\
        Max-cut & \textbf{0.610} & 0.585 & 0.190 & 0.931 & 1.00 \\
        SK model & \textbf{0.740} & 0.538 & 0.230 & 0.954 & 1.00 \\ \midrule
        Average & \textbf{0.655} & 0.554 & 0.353 & 0.914 & 1.00 \\ \bottomrule
    \end{tabular}
\caption{Success probability for different solvers}
\label{results:allsuccess}
\end{table}


\chapter{NNQS Exploration}\label{nnqsresults}
This chapter explores the performance of the NNQS solver with various architectures and training schedules. All experiments in this section were run on a subset of the entire dataset with problem sizes of $10,25,50,75,200,250$ and $10$ randomly generated problems\footnote{random seed is chosen to be from $0-9$} for each problem type and size. 

%The problem evaluation metrics are also calculated by considering the solutions from the different types of NNQS to draw a clearer comparison between architectures and training schemes.

\section{Architectures and Training Schedules}
We will utilise the Restricted Boltzmann Machine (RBM) and the Multilayer Perceptron (MLP) as the underlying architecture for NNQS. For a given input problem with $n$ variables, the RBM model will have $n$ visible nodes and $5n$ hidden nodes, while the MLP will have $n$ input nodes, $1$ hidden layer of size $5n$ and $1$ positive real output node. The RBM uses the sigmoid function, while the MLP uses the ReLU activation function for the hidden layer and a sigmoid function in the output layer. We use Gibbs sampling for the RBM and Metropolis sampling for the MLP which are introduced in \autoref{samplingmethods}.

We will also compare three training schedules for the NNQS solver---progressive, direct, and continuous. The progressive training algorithm follows \autoref{alg:progressiveagain} and has been utilised in previous experiments. In progressive training, the normalised anneal fraction $s$ is incremented by $0.1$, and the NNQS is trained until convergence for up to $100$ epochs. In direct training, described in \autoref{alg:direct}, $s$ is held constant at $1$ for all $1000$ epochs. In continuous training, described in \autoref{alg:continuous}, $s$ is gradually increased every epoch. All schedules use at most $1000$ epochs.

\begin{algorithm}
    \begin{algorithmic}
    \For {$s \in [0.1, 1.0]$ step $0.1$}
    \State Set $H(s) \leftarrow A(s)\hat{H}_0 + B(s)\hat{H}_c$;
    \State Train NNQS on $H(s)$ until convergence or until epoch limit of $100$ is reached;
    \EndFor
    \end{algorithmic}
    \caption{NNQS Progressive Schedule}
    \label{alg:progressiveagain}
\end{algorithm}

\begin{algorithm}
    \begin{algorithmic}
    \State Set $H \leftarrow A(1)\hat{H}_0 + B(1)\hat{H}_c$;
    \State Train NNQS on $H$ until convergence or until epoch limit of $1000$ is reached;
    \end{algorithmic}
    \caption{NNQS Direct Schedule}
    \label{alg:direct}
\end{algorithm}

\begin{algorithm}
    \begin{algorithmic}
    \For {$s \in [0.001, 1.0]$ step $0.001$}
    \State Set $H(s) \leftarrow A(s)\hat{H}_0 + B(s)\hat{H}_c$;
    \State Train NNQS on $H(s)$ for $1$ epoch;
    \EndFor
    \end{algorithmic}
    \caption{NNQS Continuous Schedule}
    \label{alg:continuous}
\end{algorithm}

Direct training is a baseline for training a neural network with the cost function as the problem Hamiltonian for the entire training period. Progressive training most closely resembles the quantum annealing process, where the system is kept at the ground state by training until convergence after each increment of $s$. Continuous training slowly increments $s$ but does not train until convergence and is most similar to diabatic quantum annealing, where the system is allowed to "escape" to excited states during the annealing.

\section{Results and Discussion}
We present the performance metrics for each dataset, accompanied by error bars representing the unbiased standard error of the mean. Graphs with problem sizes on the x-axis are plotted with a log scale. The performance by dataset and problem size is shown in \autoref{appendix:nnqssizegraph}.

\subsection{NAE3SAT}
For the NAE3SAT dataset, the RBM with continuous training (red) performs the best for both average normalised energy and success probability when averaged across all problem sizes, shown in \autoref{nnqs-nae3sat-average}. The RBM with progressive training (brown), and the MLP with continuous training (blue) also perform well.

\begin{figure}[!htb]
    \centering
    \subfloat[Normalised energy]{\includegraphics[width=0.5\textwidth]{images/nae3sat_nnqs_avg.png}}
    \subfloat[Success probability]{\includegraphics[width=0.5\textwidth]{images/nae3sat_nnqs_success_avg.png}}
    \caption{Average performance of different NNQS types for NAE3SAT}
    \label{nnqs-nae3sat-average}
\end{figure}

\subsection{Max-cut}
For the maxcut dataset, the RBM with continuous training (red) again performs the best in terms of average normalised energy and success probability when averaged across all sizes, shown in \autoref{nnqs-maxcut-average}. Again, the RBM with progressive training (brown), and the MLP with continuous training (blue) perform well. However, the performance gap between the top three solvers is relatively small, implying that max-cut might be easier than NAE3SAT.

\begin{figure}[!htb]
    \centering
    \subfloat[Normalised energy]{\includegraphics[width=0.5\textwidth]{images/maxcut_nnqs_avg.png}}
    \subfloat[Success probability]{\includegraphics[width=0.5\textwidth]{images/maxcut_nnqs_success_avg.png}}
    \caption{Average performance of different NNQS types for max-cut}
    \label{nnqs-maxcut-average}
\end{figure}

\subsection{SK model}
For the SK model dataset, the RBM with continuous training (red) again performs the best in average normalised energy and success probability when averaged across all sizes, shown in \autoref{nnqs-skmodel-average}. The RBM with progressive training (brown) also perform well in both metrics.

\begin{figure}[!htb]
    \centering
    \subfloat[Normalised energy]{\includegraphics[width=0.5\textwidth]{images/skmodel_nnqs_avg.png}}
    \subfloat[Success probability]{\includegraphics[width=0.5\textwidth]{images/skmodel_nnqs_success_avg.png}}
    \caption{Average performance of different NNQS types for SK model}
    \label{nnqs-skmodel-average}
\end{figure}

\section{Conclusion}
\autoref{results:nnqsnormalizedenergy} and \autoref{results:nnqssuccess} summarises the average normalised energy and success probability for different types of NNQS for each dataset and also computes the average across all datasets.

Comparing the two architectures, RBM performs better than MLP in both metrics for all datasets and training schemes. In terms of training schemes, direct training performs the worst among the three, and continuous training performs slightly better than progressive training. Using the RBM with a continuous training scheme gives the best performance across all datasets. 

The RBM likely performs better as it utilises Gibbs sampling, which is a more efficient sampling method and helps the neural network approximate the wave function more closely. Gibbs sampling allows for multiple-bit flips in each iteration and is thus able to explore a larger state space.

Even though progressive training more closely mimics the quantum annealing process in D-Wave solvers, the sudden change of Hamiltonian may have led to large gradient terms that made it more difficult for the neural network to converge, leading to poorer training. Continuous training gradually changes the Hamiltonian, which limits the gradient magnitudes and may result in better training. Even though continuous training performs better, we use progressive training in previous sections as our goal is to simulate the D-Wave annealing process and we do not fully understand why continuous training performs better. Direct training is expected to perform poorly as it tends to get stuck in local minima during training. 

\begin{table}[!htb]
    \centering
    \caption{Average normalised energy for different NNQS types}
    \label{results:nnqsnormalizedenergy}
    \begin{tabular}{ccccccc} \toprule
        ~ & \multicolumn{3}{c}{MLP} & \multicolumn{3}{c}{RBM} \\
        \cmidrule{2-7} & Continuous & Direct & Progressive & Continuous & Direct & Progressive \\
        \midrule
        NAE3SAT & 0.866 & 0.118 & 0.352 & \textbf{0.997} & 0.511 & 0.910 \\
        Max-cut & 0.924 & 0.102 & 0.686 & \textbf{0.998} & 0.704 & 0.988 \\
        SK model & 0.790 & 0.248 & 0.411 & \textbf{0.999} & 0.488 & 0.995 \\ \midrule
        Average & 0.860 & 0.156 & 0.483 & \textbf{0.998} & 0.568 & 0.965 \\ \bottomrule
    \end{tabular}
\end{table}

\begin{table}[!htb]
    \centering
    \caption{Success probability for different NNQS types}
    \label{results:nnqssuccess}
    \begin{tabular}{ccccccc} \toprule
        ~ & \multicolumn{3}{c}{MLP} & \multicolumn{3}{c}{RBM} \\
        \cmidrule{2-7} & Continuous & Direct & Progressive & Continuous & Direct & Progressive \\
        \midrule
        NAE3SAT & 0.583 & 0.029 & 0.062 & \textbf{0.950} & 0.167 & 0.364 \\
        Max-cut & 0.831 & 0.000 & 0.667 & \textbf{0.965} & 0.417 & 0.892 \\
        SK model & 0.417 & 0.000 & 0.333 & \textbf{0.933} & 0.000 & 0.767 \\ \midrule
        Average & 0.610 & 0.010 & 0.354 & \textbf{0.949} & 0.194 & 0.674 \\ \bottomrule
    \end{tabular}
\end{table}
\chapter{Conclusion}

This chapter summarises our study's contributions and limitations. It also makes recommendations for further work that future projects could investigate.

\section{Contributions}
In this study, we have benchmarked $5$ QUBO solvers:
\begin{enumerate}
    \item D-Wave Quantum Annealer
    \item Neural Network Quantum States (NNQS)
    \item Quantum Approximate Optimization Algorithm (QAOA)
    \item GUROBI Optimizer
    \item Fixstars Amplify QUBO Solver
\end{enumerate}
We used three types of combinatorial optimisation problems: not-all-equal 3-satisfiability (NAE3SAT), max-cut, and Sherrington-Kirkpatrick model (SK model) problems as datasets to evaluate the solvers' performance.

Our results show that the D-Wave and NNQS solvers perform the best among the quantum-inspired solvers. The NNQS solver is generally better than the D-Wave solver except for the SK model dataset. The QAOA solver achieves generally poor performance across datasets and is not comparable due to the limitations on problem size. All three quantum-inspired solvers underperform the two classical solvers, with the simulated-annealing-based Fixstars solver achieving the best performance for all datasets.

When the runtimes of the D-Wave, GUROBI, and Fixstars solvers were matched, the D-Wave solver could outperform the GUROBI solver for specific ranges of problem sizes. This inspires further development of quantum annealers that can handle larger problems with dense connectivity.

We also investigated the NNQS solver with different architectures and training algorithms. The Restricted Boltzmann Machine (RBM) and the Multilayer Perceptron (MLP) were used as the underlying neural networks along with three different training algorithms---progressive, direct, and continuous. We found that the RBM with a continuous training scheme performs best across all datasets.

\section{Limitations and Future Work}
The primary constraint of our study came from the limitations of the QAOA solver, which was intended to be run on a gate-based quantum computer. Due to restricted availability of real quantum computers, we used a quantum simulator capable of handling only up to $30$ variables. Nevertheless, as the simulator does not model quantum noise and the results from the QAOA solver for small problems are not promising, QAOA on an real quantum computer would likely perform even worse for larger problems and is thus not too interesting to benchmark in the current NISQ era. Another limitation was that there were many parameters for each solver that we did not have the resources to optimise for, such as the annealing time for the D-Wave solver and various forms of the QAOA Ansatz. These might be more suitable for future projects that investigates one specific QUBO solver.

Future studies can explore more QUBO problem types and attempt to classify problems that are difficult for annealing-based solvers, such as quantum annealers and simulated annealers, but easier for other solvers like the QAOA solver. When gate-based quantum computers are readily available, the QAOA solver with larger $p$ values ($p > 1$) could also be benchmarked, which has been shown to perform better but requires more computational resources.

For further work with NNQS, future studies could investigate if more modern machine learning models, such as Graph Neural Networks or Attention-based Neural Networks, can be used as the underlying architecture for NNQS and whether they provide better performance. It would also be interesting to investigate why the continuous training scheme performs better than the progressive training scheme.

Future studies could also determine whether NNQS closely approximates the wave function of a D-Wave solver during the quantum annealing process. A quench of the D-Wave annealing process can help take a snapshot of the intermediate state, which can be compared to the intermediate sampling results from the NNQS solver. Extra information is included in \autoref{appendix:quenching}.

\bibliographystyle{socreport}
\bibliography{report}

% Appendix (Remove if no appendix)
\appendix
\chapter{Reformulating the knapsack problem as a QUBO problem}\label{appendix:knapsack}

This appendix explains how the knapsack problem, a combinatorial optimisation problem with inequality constraints, can be formalised as a QUBO problem. The knapsack problem is a classic optimisation problem that consists of a knapsack that has an integer weight limit $W > 0$ and $n$ items, each with a weight $w_i > 0$ and profit $c_i > 0$. The problem seeks to find the optimal set of items to place in the knapsack to maximise total profit while not exceeding the weight limit. Formally, we have
\begin{align}
\max &\sum_{i=1}^n c_i x_i \label{eq:knapsack_cost}\\
\mathrm{st.} &\sum_{i=1}^n w_i x_i \leq W \label{eq:knapsack_constraint}\\
x_i &\in \{0,1\}^n \nonumber
\end{align}
The quantity \refeq{eq:knapsack_cost} refers to the total profit of the chosen items, and constraint \refeq{eq:knapsack_constraint} keeps the total weight below the weight limit. To convert this optimisation problem into a QUBO problem, we can include the inequality constraint in our QUBO formulation by introducing slack variables, $y_j$'s, which turn the inequality constraint into an equality constraint~\cite{b6}.
\begin{align}
\max &\sum_{i=1}^n c_i x_i \\
\mathrm{st.} &\sum_{i=1}^n w_i x_i = \sum_{j=1}^W jy_j \label{eq:knapsack_slack_weight}\\
&\sum_{j=1}^W y_j = 1 \label{eq:knapsack_slack_sum}\\
x &\in \{0,1\}^n, y \in \{0,1\}^W \nonumber
\end{align}
The slack variables $y_j$ track the total weight of the knapsack with $y_j = 1 \Leftrightarrow$ total weight is $j$. Constraint \refeq{eq:knapsack_slack_sum} ensures that only one of $y_j$ equals $1$. With penalty parameters $C_1, C_2 > \sum_{i=1}^n c_i$, we can formulate the following QUBO problem:
\begin{align}
    \max f(x, y) &\coloneqq \sum_{i=1}^n c_i x_i - C_1 P_w^2 - C_2 P_n^2 \\
    P_w &\coloneqq \sum_{i=1}^n w_i x_i - \sum_{j=1}^W jy_j \\
    P_n &\coloneqq 1 - \sum_{j=1}^W y_j \\
    x &\in \{0,1\}^n, y \in \{0,1\}^W \nonumber
\end{align}
Since the penalty parameters are larger than $\sum_{i=1}^n c_i$, the optimal solution to the QUBO problem must have $P_w = P_n = 0$. Hence, it ensures that exactly one of the $y_i=1$ and that the total weight is below the knapsack capacity. Since the optimal solution to the original knapsack must also have a total weight below the knapsack capacity, the value of $x$ that solves the QUBO must also solve the original knapsack problem. We can also formulate this QUBO problem in terms of $Q$ by following similar steps as \autoref{subsection:example_qubo}.
\chapter{Curve fitting for NNQS}\label{appendix:curvefitting}
This section will describe how we obtained an analytical form of functions $A(s)$ and $B(s)$ as shown in \autoref{dwaveannealing}. The exact form is not provided in the D-Wave documentation, which only provides a set of $1000$ discrete points \cite{dwavefunctions}. A snapshot of the discrete points is shown in \ref{tab:dwavefunction}.

\begin{table}[!h]
    \centering
    \caption{Discrete points of annealing functions $A(s)$ and $B(s)$}
    \label{tab:dwavefunction}
    \begin{tabular}{cccc}
    \hline
    $s$ & $A(s)$ (GHz) & $B(s)$ (GHz)\\
    \hline
    0 & 9.839417 & 0.259193 \\
    0.001 & 9.746483 & 0.261647 \\
    0.002 & 9.655434 & 0.264113 \\
    ... & ... & ... \\
    0.998 & 0.000000 & 8.449086 \\
    0.999 & 0.000000 & 8.463072 \\
    1.000 & 0.000000 & 8.477070 \\
    \hline
    \end{tabular}
\end{table}

To obtain an analytical form, we utilised the curve fit function from the scipy library. First, we normalise by the maximum value in $B(s)$ as only the ratio of $A$ and $B$ is important. We fit an exponential decay function $a_{A}\cdot e^{-b_{A}\cdot s} + c_{A}$ to $A(s)$ and a quadratic function $a_{B} \cdot s^2 + b_{B} \cdot s + c_{B}$ to $B(s)$. The forms of the functions were chosen with reference to the D-Wave documentation \cite{dwavefunctions}. We obtain fitted constants of $a_{A}=1.11, b_{A} = 7.06, c_A-0.00569, a_B = 0.680, b_B = 0.288, c_B = 0.0305$. The fitted functions are shown in \autoref{fittedequation}.
\begin{figure}[!h]
    \centering
    \includegraphics[width=0.6\linewidth]{images/fitted.jpg}
    \caption{Fitted $A(s)$ and $B(s)$ equations against the normalised annealing fraction $s$}
    \label{fittedequation}
\end{figure}
\chapter{NNQS exploration performance by sizes}\label{appendix:nnqssizegraph}

\section{NAE3SAT}
Refer to \autoref{nnqs-nae3sat-size}.

\begin{figure}[!htbp]
    \centering
    \subfloat[Normalized energy]{\includegraphics[width=0.9\textwidth]{images/nae3sat_nnqs_size.png}}
    \\
    \subfloat[Success probability]{\includegraphics[width=0.9\textwidth]{images/nae3sat_nnqs_success_size.png}}
    \caption{Performance of different NNQS types for NAE3SAT by problem size}
    \label{nnqs-nae3sat-size}
\end{figure}

\section{Max-cut}
Refer to \autoref{nnqs-maxcut-size}.

\begin{figure}[!htbp]
    \centering
    \subfloat[Normalized energy]{\includegraphics[width=0.9\textwidth]{images/maxcut_nnqs_size.png}}
    \\
    \subfloat[Success probability]{\includegraphics[width=0.9\textwidth]{images/maxcut_nnqs_success_size.png}}
    \caption{Performance of different NNQS types for max-cut by problem size}
    \label{nnqs-maxcut-size}
\end{figure}

\section{SK model}
Refer to \autoref{nnqs-skmodel-size}.

\begin{figure}[!htbp]
    \centering
    \subfloat[Normalized energy]{\includegraphics[width=0.9\textwidth]{images/skmodel_nnqs_size.png}}%\hfill
    \\
    \subfloat[Success probability]{\includegraphics[width=0.9\textwidth]{images/skmodel_nnqs_success_size.png}}
    \caption{Performance of different NNQS types for SK model by problem size}
    \label{nnqs-skmodel-size}
\end{figure}
\chapter{Average runtime of solvers by problem type and sizes}\label{appendix:timesizegraph}

\section{NAE3SAT}
\autoref{time-nae3sat-size} shows the average runtime taken for NAE3SAT problems.
\begin{figure}[!htb]
    \centering
    \includegraphics[width=0.6\textwidth]{images/nae3sat_all_time_size.png}
    \caption{Average runtime taken by different solvers for NAE3SAT by problem size}
    \label{time-nae3sat-size}
\end{figure}

\section{Max-cut}
\autoref{time-maxcut-size} shows the average runtime taken for max-cut problems.
\begin{figure}[!htb]
    \centering
    \includegraphics[width=0.6\textwidth]{images/maxcut_all_time_size.png}
    \caption{Average runtime taken by different solvers for max-cut by problem size}\label{time-maxcut-size}
\end{figure}

\section{SK model}
\autoref{time-skmodel-size} shows the average runtime taken for SK model problems.
\begin{figure}[!htb]
    \centering
    \includegraphics[width=0.6\textwidth]{images/skmodel_all_time_size.png}
    \caption{Average runtime taken by different solvers for SK model by problem size}    \label{time-skmodel-size}
\end{figure}
\chapter{Metadata of solvers by problem type and sizes}\label{appendix:metadata}

\section{D-wave}
For the D-wave solver, we recorded the average number of embedded qubits, which is needed during the minor embedding phase shown in \autoref{table:numberofembedded}.

\begin{table}[!ht]
    \centering
    \resizebox{\textwidth}{!}{%
    \begin{tabular}{lrrrrrrrrrrrrr} \toprule
        $n$ & 10 & 15 & 20 & 25 & 30 & 35 & 50 & 75 & 100 & 150 & 200 & 250 & 300 \\ \midrule
        NAE3SAT & 15.15 & 27.90 & 42.20 & 58.85 & 80.90 & 102.25 & 184.05 & 379.65 & 622.15 & 1381.00 & 2322.65 & 3756.20 & 4493.60 \\
        Max-cut & 12.70 & 25.95 & 45.45 & 69.90 & 102.60 & 142.50 & 304.05 & 706.05 & 1290.00 & 3087.95 & - & - & - \\
        SK model & 17.05 & 35.80 & 54.50 & 87.65 & 121.15 & 169.80 & 331.60 & 739.45 & 1338.65 & 2995.25 & - & - & - \\ \bottomrule
    \end{tabular}}
    \caption{Average number of embedded qubits for the D-wave solver by problem type and size}
    \label{table:numberofembedded}
\end{table}

\section{QAOA}
For the QAOA solver, we recorded the average number of quantum gates and the circuit depth shown in \autoref{table:gates} and \autoref{table:depth}, which help to quantify the complexity of the quantum circuit.

\begin{table}[!ht]
    \centering
    \begin{tabular}{lrrrrr} \toprule
        $n$ & 10 & 15 & 20 & 25 & 30\\ \midrule
        NAE3SAT & 77.75 & 127.80 & 181.50 & 235.20 & 292.30 \\
        Max-cut & 75.00 & 131.00 & 200.00 & 281.00 & 375.00 \\
        SK model & 95.00 & 180.00 & 290.00 & 425.00 & 585.00\\ \bottomrule
    \end{tabular}
    \caption{Average number of quantum gates in the quantum circuit used by the QAOA solver by problem type and size}
    \label{table:gates}
\end{table}

\begin{table}[!ht]
    \centering
    \begin{tabular}{lrrrrr} \toprule
        $n$ & 10 & 15 & 20 & 25 & 30\\ \midrule
        NAE3SAT & 18.65 & 25.30 & 30.70 & 35.15 & 40.00 \\
        Max-cut & 19.00 & 28.05 & 36.50 & 45.60 & 55.05 \\
        SK model & 22.00 & 32.00 & 42.00 & 52.00 & 62.00\\ \bottomrule
    \end{tabular}
    \caption{Average depth of the quantum circuit used by the QAOA solver by problem type and size}
    \label{table:depth}
\end{table}
\chapter{D-wave Quenching}\label{appendix:quenching}
To be filled in


\end{document}
