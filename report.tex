\documentclass[fyp,12pt]{socreport}

% Some generic packets.
\usepackage{color, colortbl}
\usepackage{url}
\usepackage{graphicx}
\usepackage{caption}
\usepackage{subcaption}
\usepackage{pgfplots}
\usepackage{tabularx}
\usepackage{multirow}
\usepackage{listings}
\usepackage{fullpage}
\pgfplotsset{width=10cm,compat=1.9}

% Sets the root path to look for all images.
\graphicspath{{images/}}

% Sets default options for listings.
\renewcommand\lstlistlistingname{List of Listings}
\newcommand*\lstinputpath[1]{\lstset{inputpath=#1}}
\lstinputpath{listings}
\lstset{frame=single, tabsize=2, captionpos=b}
\newcommand{\itab}[1]{\hspace{0em}\rlap{#1}}
\newcommand{\tab}[1]{\hspace{.11\textwidth}\rlap{#1}}

\begin{document}
\pagenumbering{roman}

% Replace as necessary
\title{Development of a Database Link Between Mainframe and PC}
\author{Chua Meng Lee}
\projyear{AY 2016/2017}
\projnumber{H90000}
\advisor{Assoc Prof Jarzabek Stanislaw}
\deliverables{
    \item \itab{Report:} \tab{1 Volume}
    \item \itab{Manual:} \tab{1 Volume}
    \item \itab{Program:} \tab{1 Diskette}
    \item \itab{Database:} \tab{1 Diskette}
}
\maketitle

% Replace as necessary
\begin{abstract}
The use of Wireless Sensor Networks for environmental monitoring has become
increasingly popular over the past decade due to its affordability, ease of deployment
and customisation, as well as its potentiality in the processing of sensed data. One of the
greatest challenges in this field would be in the design and implementation of an
efficient routing protocol which takes into account the various limitations of Wireless
Sensor Networks, such as battery life, limited storage capacities and high probability of
packet losses. Besides this, it is also extremely difficult to evaluate the performance of
such a protocol under crisis scenarios, due to its infrequency and unpredictability. In our
work, we have designed a routing protocol based on optimised Virtual Polar Coordinate
Routing (VPCR) (Newsome and Song, 2003) for use with our three-dimensional
testbed, comprising of 48 MICAz (Crossbow) motes spread across two floors of a
building. We have also developed a Java-based application with features for Event
Emulation and simple nodal analysis to assist us in our experiments. The overall
performance of our protocol will be gauged based on the average Path Stretch Factor
and path length comparisons between optimised and naïve VPCR.

\begin{descriptors}
    \item \itab{C.2.1}	\tab{Network Architecture and Design}
    \item \itab{C.2.2}	\tab{Network Protocols}
    \item \itab{C.2.4}	\tab{Distributed Systems}
    \item \itab{C.4} 	\tab{Performance of Systems}
    \item \itab{I.2.9}	\tab{Robotics}
\end{descriptors}
\begin{keywords}
    Wireless communication, routing protocols, distributed applications, fault tolerance, sensors
\end{keywords}

% Replace/Delete as necessary
\begin{implement}
    Ubuntu Linux 7.04 Feisty Fawn, TinyOS 2.x, NesC 1.2.8a, Java 1.6 SE, Xbow
Motes, Tembusu cluster
\end{implement}

\end{abstract}

% Replace as necessary
\begin{acknowledgement}
See \texttt{README.md} of this repository.
\end{acknowledgement}

\listoffigures % Remove if no figures
\listoftables % Remove if no tables
\lstlistoflistings % Remove if no listings
\tableofcontents

% Actual contents in the contents folder.
% Include additional tex files here.
\SetPicSubDir{ch-Intro}

\chapter{Introduction}
\vspace{2em}

\section{Motivation}
The quadratic unconstrained binary optimization (QUBO) problem arises in many types of combinatorial optimization tasks and has vast applications in both operational research and industry~\cite{b1}. Problems that involve choosing a set of binary decisions to optimise an objective function can usually be expressed as a QUBO problem. However, QUBO problems are NP-hard and are extremely hard to solve efficiently~\cite{barahona1982computational,b1}. Traditionally, QUBO-solving methods are specialized for a specific problem domain in order to leverage the unique characteristics of each problem domain, limiting the versatility and portability of QUBO-solving methods~\cite{b5}.

The QUBO problem shares similarities with the Ising model in Physics, where variables are expressed as either $-1$ or $1$, instead of the binary variables used in QUBO. The two models turn out to be equivalent and the correspondence between the QUBO problem and the Ising model has opened the doors to solving QUBO problems with quantum computational methods~\cite{b5}. The equivalence not only enhances the problem domains that could be modelled as QUBO problems but also allows for the QUBO problem to be solved by more general quantum-based solving methods~\cite{b5}.

With a broad range of classical and quantum-based methods available to tackle QUBO problems, it is imperative to consolidate and benchmark existing QUBO-solving methods to highlight their strengths and weaknesses for solving different kinds of QUBO problems.

\section{Objectives}
In the first section of the report, we aim to measure the performance of 3 quantum-based QUBO-solving methods:
\begin{itemize}
    \item Quantum Annealing
    \item Neural Network Quantum States
    \item Quantum Approximate Optimization Algorithm
\end{itemize}
For comparison with quantum-based methods, we will also use classical solvers as a baseline. We will explore 2 classical solvers:
\begin{itemize}
    \item Fixstars Amplify QUBO Solver
    \item GUROBI Optimizer
\end{itemize}

%QUBOs are not always effective in practice, though. In particular, they cannot directly:

%Support common real-world situations, such as continuous variables (e.g., prices, commodity flow)
%Support constraints (e.g., max/mins, price cannot exceed a certain value)
%Prove optimality (e.g., you can’t be sure you’ve identified the best solution)

For each solver, we will measure the success rate (probability of solving optimally) and effectiveness (performance of candidate solution) across QUBO problems of different types and sizes. More details of the solvers and performance metrics are available in \autoref{review} and \autoref{methodology}. The results of the study will allow us to understand the current state of quantum-based QUBO-solving methods and how they compare to classical solvers. 

In the second section of the report, we will further explore how Neural Network Quantum States can be used to simulate the quantum annealing process to solve QUBO problems. We first sample intermediate states from both the NNQS and a physical quantum annealer to compare the annealing process that is being simulated in the NNQS. We will also conduct a systematic comparison of two different training algorithms highlighted in \autoref{methodology} for Neural Network Quantum States and extract any physical insights they provide which will bring us one step closer in harnessing its full potential.

\chapter{Benchmarking QUBO solvers}\label{benchmark}
This chapter first introduces related benchmarking work for QUBO problems and QUBO solvers. Then, we present our results for the solvers and datasets used for this study.

\section{Related benchmarking work}
One of the first benchmarking works for quantum annealing was conducted by Denchev et al. \yrcite{denchev2016computational}, who measured the performance of D-Wave quantum annealing on the older D-Wave 2X machine using specially crafted problems that have tall and narrow energy barriers separating local minima. Quantum annealing is expected to be $1.8 \times 10^8$ times faster than simulated annealing, which tends to fail with problems with such an energy landscape.


\outcite{b34} evaluated the performance of QAOA on the IBMQ backend and the D-Wave solver using instances of MaxCut and 2-satisfiability problems with up to 18 variables. The performance of the QAOA algorithm is inconsistent and underperforms quantum annealing in their problem set. More recently, \outcite{b35} also compared the performance of QAOA on the IBMQ backend and D-Wave quantum annealing on randomly generated Ising problems with cubic interaction terms and found that quantum annealing had superior performance over QAOA for all problem sizes.

\outcite{gomes2019classical} showed that the NNQS solving method with an RBM architecture produces good quality solutions for the max-cut problem with graph sizes of up to 256. \outcite{khandoker2023supplementing} uses recurrent neural networks as the NNQS architecture for the max-cut and travelling salesman problem and found that it outperforms SA. However, no direct study has compared performance across quantum annealing, QAOA, and NNQS.

\section{Results and Discussion}
Performance is shown for each dataset, accompanied by error bars representing each data point's unbiased standard error of the mean. Graphs with problem sizes on the x-axis are plotted with a log scale.

\subsection{NAE3SAT}

\begin{figure}[!htbp]
    \centering
    \subfloat[Normalized energy]{\includegraphics[width=0.9\textwidth]{images/nae3sat_all_size.png}}
    \\
    \subfloat[Success probability]{\includegraphics[width=0.9\textwidth]{images/nae3sat_all_success_size.png}}
    \caption{Performance of different solvers for NAE3SAT by problem size}
    \label{all-nae3sat-size}
\end{figure}

\begin{figure}[!htbp]
    \centering
    \subfloat[Normalized energy]{\includegraphics[width=0.49\textwidth]{images/nae3sat_all_avg.png}}\hfill
    \subfloat[Success probability]{\includegraphics[width=0.49\textwidth]{images/nae3sat_all_success_avg.png}}
    \caption{Average performance of different solvers for NAE3SAT}
    \label{all-nae3sat-average}
\end{figure}

Performance by size for the NAE3SAT dataset in \autoref{all-nae3sat-size} and average performance is shown in \autoref{all-nae3sat-average}. The D-wave solver and NNQS could both solve problems up to $n=300$. However, multiple embedding requests were required for problems of size $300$ for the D-wave Pegasus topology, which suggests that $n=300$ might be near the D-wave size limit for the NAE3SAT problem. QAOA could only solve problems of up to $n=30$ due to the limitations on the number of qubits of the simulator.

In terms of performance, the D-wave solver performs well up to $n=50$ with a success probability of $1$. For larger problem sizes, the performance of the D-wave solver drops off sharply. The NNQS performs well up to $n=20$, with the success probability and normalised energy gradually decreasing until $n=300$. The QAOA solver performs well up to $n=15$, and performance decreases until $n=30$. Between the classical solvers, the Fixstars QUBO solver performs better than the GUROBI optimiser at larger problem sizes ($>150$). Both classical solvers outperform the quantum-inspired solvers.

Overall, the NNQS has the highest average normalised energy among the three quantum-inspired solvers but has the lowest success probability, which is likely due to it being able to solve problems of larger sizes that the D-wave Annealer and QAOA solver cannot handle. The QAOA solver has the highest success probability but can only handle problems up to $n=30$.

\subsection{Max-cut}

\begin{figure}[!htbp]
    \centering
    \subfloat[Normalized energy]{\includegraphics[width=0.9\textwidth]{images/maxcut_all_size.png}}
    \\
    \subfloat[Success probability]{\includegraphics[width=0.9\textwidth]{images/maxcut_all_success_size.png}}
    \caption{Performance of different solvers for max-cut by problem size}
    \label{all-maxcut-size}
\end{figure}

\begin{figure}[!htbp]
    \centering
    \subfloat[Normalized energy]{\includegraphics[width=0.49\textwidth]{images/maxcut_all_avg.png}}\hfill
    \subfloat[Success probability]{\includegraphics[width=0.49\textwidth]{images/maxcut_all_success_avg.png}}
    \caption{Average performance of different solvers for max-cut}
    \label{all-maxcut-average}
\end{figure}

Performance by size for the max-cut dataset is shown in \autoref{all-maxcut-size}, and average performance is shown in \autoref{all-maxcut-average}. For the max-cut problem, the D-wave solver could only handle problem sizes up to $n=150$ due to the need for minor embedding onto the pegasus topology. QAOA solved problems of up to $n=30$ due to the limitations of the simulator.

In terms of performance, the D-wave solver performs well up to $n=30$, and performance drops off sharply for larger problems. The NNQS performs well at $n=150$, although the success probability decreases for problem sizes over $50$. The QAOA solver performs well only for $n=10$, with performance decreasing for larger problems.

Overall, the NNQS has the highest average normalised energy among the three quantum-inspired solvers and has a slightly lower success probability than the D-wave solver. The QAOA solver performs poorly in both metrics.

\subsection{SK model}

\begin{figure}[!htbp]
    \centering
    \subfloat[Normalized energy]{\includegraphics[width=0.9\textwidth]{images/skmodel_all_size.png}}%\hfill
    \\
    \subfloat[Success probability]{\includegraphics[width=0.9\textwidth]{images/skmodel_all_success_size.png}}
    \caption{Performance of different solvers for SK model by problem size}
    \label{all-skmodel-size}
\end{figure}

\begin{figure}[!htbp]
    \centering
    \subfloat[Normalized energy]{\includegraphics[width=0.49\textwidth]{images/skmodel_all_avg.png}}\hfill
    \subfloat[Success probability]{\includegraphics[width=0.49\textwidth]{images/skmodel_all_success_avg.png}}
    \caption{Average performance of different solvers for SK model}
    \label{all-skmodel-average}
\end{figure}

Performance by size for the SK model dataset is shown in \autoref{all-skmodel-size}, and average performance is shown in \autoref{all-skmodel-average}. For the SK model, the D-wave solver could only handle problem sizes up to $n=150$ due to the need for minor embedding onto the pegasus topology. The SK model is fully connected, which makes embedding difficult for the D-wave QPU. QAOA solved problems of up to $n=30$ due to the limitations of the simulator.

Due to its multi-valley energy landscape, the SK model presents a difficult problem for all QUBO solvers. The D-wave solver performs well up to $n=30$, gradually decreasing performance for larger problems. The NNQS follows a similar trend, although it can solve problems of larger sizes and performs better at sizes of $=100,150$. The QAOA solver has consistently poor performance across problem sizes.

Overall, the D-wave annealer has the highest average normalised energy and success probability among the three quantum-inspired solvers. The NNQS is slightly worse in both metrics, while the QAOA solver performs poorly for both metrics.

\section{Time-Constrained Solver Comparison}
We also measured the runtime for each solver for each problem and the average runtime across all problems with the same size $n$ shown in \autoref{results:timeaverage}. Runtimes split by problem type can be found in \autoref{appendix:timesizegraph}

For the D-wave solver, the average runtime increases approximately linearly from $0.128\si{\second}$ for $n=10$ to $0.184\si{\second}$ for $n=50$ and $0.292\si{\second}$ for $n=300$. 

For the NNQS solver, the runtime does not increase significantly from $n=10$ to $n=150$ and remains around from $250\si{\second}$ to $350\si{\second}$ but increases sharply for larger $n \geq 200$ which could be due to memory issues with the GPU or the sampling may have taken a longer time to converge. 

The QAOA solver's runtime remains stable from $n=10$ to $n=20$ at around $38 \si{\second}$. However, it increases rapidly to $6847 \si{\second}$ at $n=30$ due to the need for more optimisation iterations and greater computational resources for the simulation.

The GUROBI optimiser was set with a maximum time limit of $600 \si{\second}$ but could finish solving before the time limit for most problems with $n \leq 35$. The Fixstar solver was configured with a maximum time limit of $100 \si{\second}$, the maximum possible duration, and each solve utilised the entire time limit.

\begin{figure}[!htbp]
    \centering
    \includegraphics[width=0.9\textwidth]{images/all_time_average.png}
    \caption{Average runtime in log scale taken by different solvers for QUBO problems by size}
    \label{results:timeaverage}
\end{figure}

Using the average runtime of the D-wave solver, we conducted a second run of the benchmarking to measure the performance of the D-wave solver against the classical solvers---GUROBI and Fixstar. This experiment aimed to test if the D-wave solver could outperform the classical solvers if they were limited to the same runtime. For a problem of a specific type and size, the classical solvers were run with a maximum runtime equal to the average time required by the D-wave solver for problems of equivalent type and size shown in \autoref{results:averageruntimedwave}.

\begin{table}[!ht]
    \centering
    \resizebox{\textwidth}{!}{%
    \begin{tabular}{lrrrrrrrrrrrrr} \toprule
        $n$ & 10 & 15 & 20 & 25 & 30 & 35 & 50 & 75 & 100 & 150 & 200 & 250 & 300 \\ \midrule
        NAE3SAT & 0.133 & 0.135 & 0.141 & 0.149 & 0.138 & 0.156 & 0.173 & 0.184 & 0.201 & 0.243 & 0.259 & 0.286 & 0.292 \\
        Max-cut & 0.127 & 0.136 & 0.1560 & 0.130 & 0.155 & 0.160 & 0.183 & 0.223 & 0.247 & 0.278 & - & - & - \\
        SK model & 0.125 & 0.137 & 0.140 & 0.140 & 0.150 & 0.161 & 0.195 & 0.221 & 0.245 & 0.278 & - & - & - \\ \bottomrule
    \end{tabular}}
    \caption{Average runtime (seconds) of the D-wave solver by problem type and size. Dashes indicate that the D-wave solver could not embed problems of that size.}
    \label{results:averageruntimedwave}
\end{table}

The results are shown in \autoref{all-time-size}. For each problem type, the performance of the GUROBI solver drops before the D-wave solver, and there are problem types and sizes where the D-wave solver outperforms the GUROBI solver when the runtime is matched. The D-wave solver outperforms GUROBI for NAE3SAT with $n=50, 75$, max-cut with $n=30,35$, and SK model with $n=20, 35$. However, when the problem sizes increase even further, the D-wave solver performs poorly, possibly due to the increased noise of the quantum annealer. The Fixstar solver remains the top-performing solver across all problem types and sizes, even when the runtime is matched with the D-wave solver. The results show that the D-wave solver can outperform classical solvers like GUROBI for specific problem sizes when the runtime is matched.

\begin{figure}[!htbp]
    \centering
    \subfloat[NAE3SAT]{\includegraphics[width=0.7\textwidth]{images/nae3sat_timing_size.png}}%\hfill
    \\
    \subfloat[Max-cut]{\includegraphics[width=0.7\textwidth]{images/maxcut_timing_size.png}}
    \\    
    \subfloat[SK model]{\includegraphics[width=0.7\textwidth]{images/skmodel_timing_size.png}}
    \caption{Performance of D-wave solver against GUROBI and Fixstar by problem type and size}
    \label{all-time-size}
\end{figure}

\section{Solver logs}
During the benchmarking experiments, supplementary metadata from the D-wave and QAOA solver was recorded. We recorded the embedded qubit count for the D-wave solver, which tends to grow differently for different problem types. For the QAOA solver, we recorded the number of quantum gates and the circuit depth, which can help quantify the complexity of the quantum circuit used. A metadata summary is available in \autoref{appendix:metadata}.

\section{Conclusion}
When comparing the three quantum-inspired solvers, NNQS has the best normalised energy for the NAE3SAT and max-cut dataset, while the D-wave solver has the best normalised energy for the SK model performance. NNQS also does the best in normalised energy when averaged across the three datasets. NNQS tends to do better in normalised energy since it minimises the energy expectation value, which optimises the average energy of samples but does not necessarily optimise for the highest probability of sampling the best solution. 

QAOA has the highest success probability for the NAE3SAT dataset. However, it is important to note that it could only handle problems with up to $30$ variables. The D-wave solver has the best success probability for the max-cut and SK model datasets and the average success probability across the three datasets.

\autoref{results:allnormalizedenergy} and \autoref{results:allsuccess} show the average normalised energy and success probability for different solvers for each dataset and the average across all datasets. Across all datasets and both metrics, the Fixstar QUBO solver has the best performance, consistently returning the best solutions out of all solvers.


\begin{table}[!ht]
    \centering
    \begin{tabular}{cccccc} \toprule
        ~ & D-wave & NNQS & QAOA & GUROBI & Fixstar \\ \midrule
        NAE3SAT & 0.617 & \textbf{0.755} & 0.640 & 0.983 & 1.00 \\
        Max-cut & 0.610 & \textbf{0.753} & 0.190 & 0.991 & 1.00 \\
        SK model & \textbf{0.761} & 0.664 & 0.230 & 0.990 & 1.00 \\ \midrule
        Average & 0.663 & \textbf{0.724} & 0.353 & 0.988 & 1.00 \\ \bottomrule
    \end{tabular}
    \caption{Average normalised energy for different solvers}
    \label{results:allnormalizedenergy}
\end{table}

\begin{table}[!ht]
    \centering
    \begin{tabular}{cccccc} \toprule
        ~ & D-wave & NNQS & QAOA & GUROBI & Fixstar \\ \midrule
        NAE3SAT & 0.615 & 0.538 & \textbf{0.640} & 0.858 & 1.00 \\
        Max-cut & \textbf{0.610} & 0.585 & 0.190 & 0.931 & 1.00 \\
        SK model & \textbf{0.740} & 0.538 & 0.230 & 0.954 & 1.00 \\ \midrule
        Average & \textbf{0.655} & 0.554 & 0.353 & 0.914 & 1.00 \\ \bottomrule
    \end{tabular}
\caption{Success probability for different solvers}
\label{results:allsuccess}
\end{table}


\chapter{Conclusion}

This chapter summarises our study's contributions and limitations. It also makes recommendations for further work that future projects could investigate.

\section{Contributions}
In this study, we have benchmarked $5$ QUBO solvers:
\begin{enumerate}
    \item D-Wave Quantum Annealer
    \item Neural Network Quantum States (NNQS)
    \item Quantum Approximate Optimization Algorithm (QAOA)
    \item GUROBI Optimizer
    \item Fixstars Amplify QUBO Solver
\end{enumerate}
We used three types of combinatorial optimisation problems: not-all-equal 3-satisfiability (NAE3SAT), max-cut, and Sherrington-Kirkpatrick model (SK model) problems as datasets to evaluate the solvers' performance.

Our results show that the D-Wave and NNQS solvers perform the best among the quantum-inspired solvers. The NNQS solver is generally better than the D-Wave solver except for the SK model dataset. The QAOA solver achieves generally poor performance across datasets and is not comparable due to the limitations on problem size. All three quantum-inspired solvers underperform the two classical solvers, with the simulated-annealing-based Fixstars solver achieving the best performance for all datasets.

When the runtimes of the D-Wave, GUROBI, and Fixstars solvers were matched, the D-Wave solver could outperform the GUROBI solver for specific ranges of problem sizes. This inspires further development of quantum annealers that can handle larger problems with dense connectivity.

We also investigated the NNQS solver with different architectures and training algorithms. The Restricted Boltzmann Machine (RBM) and the Multilayer Perceptron (MLP) were used as the underlying neural networks along with three different training algorithms---progressive, direct, and continuous. We found that the RBM with a continuous training scheme performs best across all datasets.

\section{Limitations and Future Work}
The primary constraint of our study came from the limitations of the QAOA solver, which was intended to be run on a gate-based quantum computer. Due to restricted availability of real quantum computers, we used a quantum simulator capable of handling only up to $30$ variables. Nevertheless, as the simulator does not model quantum noise and the results from the QAOA solver for small problems are not promising, QAOA on an real quantum computer would likely perform even worse for larger problems and is thus not too interesting to benchmark in the current NISQ era. Another limitation was that there were many parameters for each solver that we did not have the resources to optimise for, such as the annealing time for the D-Wave solver and various forms of the QAOA Ansatz. These might be more suitable for future projects that investigates one specific QUBO solver.

Future studies can explore more QUBO problem types and attempt to classify problems that are difficult for annealing-based solvers, such as quantum annealers and simulated annealers, but easier for other solvers like the QAOA solver. When gate-based quantum computers are readily available, the QAOA solver with larger $p$ values ($p > 1$) could also be benchmarked, which has been shown to perform better but requires more computational resources.

For further work with NNQS, future studies could investigate if more modern machine learning models, such as Graph Neural Networks or Attention-based Neural Networks, can be used as the underlying architecture for NNQS and whether they provide better performance. It would also be interesting to investigate why the continuous training scheme performs better than the progressive training scheme.

Future studies could also determine whether NNQS closely approximates the wave function of a D-Wave solver during the quantum annealing process. A quench of the D-Wave annealing process can help take a snapshot of the intermediate state, which can be compared to the intermediate sampling results from the NNQS solver. Extra information is included in \autoref{appendix:quenching}.

\bibliographystyle{socreport}
\bibliography{report}

% Appendix (Remove if no appendix)
\appendix
\chapter{Curve fitting for NNQS}\label{appendix:curvefitting}
This section will describe how we obtained an analytical form of functions $A(s)$ and $B(s)$ as shown in \autoref{dwaveannealing}. The exact form is not provided in the D-Wave documentation, which only provides a set of $1000$ discrete points \cite{dwavefunctions}. A snapshot of the discrete points is shown in \ref{tab:dwavefunction}.

\begin{table}[!h]
    \centering
    \caption{Discrete points of annealing functions $A(s)$ and $B(s)$}
    \label{tab:dwavefunction}
    \begin{tabular}{cccc}
    \hline
    $s$ & $A(s)$ (GHz) & $B(s)$ (GHz)\\
    \hline
    0 & 9.839417 & 0.259193 \\
    0.001 & 9.746483 & 0.261647 \\
    0.002 & 9.655434 & 0.264113 \\
    ... & ... & ... \\
    0.998 & 0.000000 & 8.449086 \\
    0.999 & 0.000000 & 8.463072 \\
    1.000 & 0.000000 & 8.477070 \\
    \hline
    \end{tabular}
\end{table}

To obtain an analytical form, we utilised the curve fit function from the scipy library. First, we normalise by the maximum value in $B(s)$ as only the ratio of $A$ and $B$ is important. We fit an exponential decay function $a_{A}\cdot e^{-b_{A}\cdot s} + c_{A}$ to $A(s)$ and a quadratic function $a_{B} \cdot s^2 + b_{B} \cdot s + c_{B}$ to $B(s)$. The forms of the functions were chosen with reference to the D-Wave documentation \cite{dwavefunctions}. We obtain fitted constants of $a_{A}=1.11, b_{A} = 7.06, c_A-0.00569, a_B = 0.680, b_B = 0.288, c_B = 0.0305$. The fitted functions are shown in \autoref{fittedequation}.
\begin{figure}[!h]
    \centering
    \includegraphics[width=0.6\linewidth]{images/fitted.jpg}
    \caption{Fitted $A(s)$ and $B(s)$ equations against the normalised annealing fraction $s$}
    \label{fittedequation}
\end{figure}
\chapter{D-wave Quenching}\label{appendix:quenching}
To be filled in

\end{document}
